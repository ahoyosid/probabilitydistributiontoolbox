\documentclass[11pt]{article}
% Bibliographies (inline)
\usepackage{filecontents}
\def\withindex{0}
\def\withnotes{0}
\def\withcolors{0}

\newcommand{\insertref}[1]{\todo[color=green!40]{#1}\xspace}
\newcommand{\explainindetail}[1]{\todo[color=red!30]{#1}\xspace}
\newcommand{\inlinetodo}[1]{\todo[color=blue!25,inline]{#1}\xspace}

\usepackage[T1]{fontenc}
\usepackage[utf8]{inputenc}

%% Eye-candy
\usepackage{lmodern}
\usepackage{xspace}                                     % Smart spacing with \xspace
\usepackage[protrusion=true,expansion=true]{microtype}  % Improve font rendering

% Striking out text
\usepackage[normalem]{ulem}

%% Math
\usepackage{amsfonts,amsmath,amssymb, amsthm, mathtools}
\usepackage{thm-restate}
\usepackage{dsfont} % For the indicator symbol

% Algorithm environment
\usepackage{algorithmicx,algpseudocode,algorithm}

% Colors (with names)
\usepackage[usenames,dvipsnames,table]{xcolor}

% Quotes: \blockquote command
\usepackage{csquotes}

% Relative sizes for text
\usepackage{relsize}

% Bibliography
%\usepackage[numbers]{natbib}

% Required for the table of results
\usepackage{multirow}
\usepackage{chngpage} % allows for temporary adjustment of side margins

% For the commands such as \capitalisewords
\usepackage{mfirstuc}

% Graphics
\usepackage{tikz}
\usetikzlibrary{arrows}
\usetikzlibrary{calc,decorations.pathmorphing,patterns}

% For indexing
\ifnum\withindex=1
  \usepackage{makeidx}
  \usepackage{ifthen}
  \newcommand\indexed[2][]{\ifthenelse{\equal{#1}{}}{#2\index{#2}}{#2\index{#1}}}
  \makeindex %%%% Enable indexing
\fi
%%\usepackage{showidx} % To debug; does not play well with hyperref

% References and links
\usepackage[backref,colorlinks,citecolor=blue,bookmarks=true,linktocpage]{hyperref}
\usepackage{aliascnt}
\usepackage[numbered]{bookmark}

% Full pages
\usepackage{fullpage}

% Titling
\usepackage{titling}

% Compressed lists
\usepackage[shortlabels]{enumitem}
  \setitemize{noitemsep,topsep=3pt,parsep=2pt,partopsep=2pt} % Uncomment for compact item lists
  \setenumerate{itemsep=1pt,topsep=2pt,parsep=2pt,partopsep=2pt}
  \setdescription{itemsep=1pt}
  
% Package for todo notes.
\ifnum\withnotes=1
  \usepackage[colorinlistoftodos,textsize=scriptsize]{todonotes}
\fi

% Verbatim inputs and code
\usepackage{verbatim}

% Resizable parentheses that work (without the space between \left(#1\right)
\usepackage{mleftright} % \mleft( #1 \mright)

\makeatletter
\@ifundefined{theorem}{%
  % Theorems (each with its own style, all same counter). Cf. http://ftp.math.purdue.edu/mirrors/ctan.org/macros/latex/contrib/hyperref/doc/manual.pdf, p.17
  \theoremstyle{plain} %% Style
  	\newtheorem{theorem}{Theorem}%[section]
  	\newaliascnt{coro}{theorem}
  	  \newtheorem{corollary}[coro]{Corollary}
  	\aliascntresetthe{coro}
  	\newaliascnt{lem}{theorem}
  		\newtheorem{lemma}[lem]{Lemma}
  	\aliascntresetthe{lem}
  	\newaliascnt{clm}{theorem}
  		\newtheorem{claim}[clm]{Claim}
	\aliascntresetthe{clm}
	\newaliascnt{fact}{theorem}
 	 	\newtheorem{fact}[theorem]{Fact}
	\aliascntresetthe{fact}
  	\newtheorem*{unnumberedfact}{Fact}
  \newaliascnt{prop}{theorem}
  		\newtheorem{proposition}[prop]{Proposition}
	\aliascntresetthe{prop}
	\newaliascnt{conj}{theorem}
  		\newtheorem{conjecture}[conj]{Conjecture}
	\aliascntresetthe{conj}
  \theoremstyle{remark} %% Style
  	\newtheorem{remark}[theorem]{Remark}
  	\newtheorem{question}[theorem]{Question}
  	\newtheorem*{notation}{Notation}
 	 \newtheorem{example}[theorem]{Example}
  \theoremstyle{definition} %% Style
  	\newaliascnt{defn}{theorem}
 		 \newtheorem{definition}[defn]{Definition}
 	 \aliascntresetthe{defn}
}{}
\makeatother
\providecommand*{\lemautorefname}{Lemma} % For \autoref{} to know the name of lemmas
\providecommand*{\clmautorefname}{Claim}
\providecommand*{\propautorefname}{Proposition}
\providecommand*{\coroautorefname}{Corollary}
\providecommand*{\defnautorefname}{Definition}
\newenvironment{proofof}[1]{\begin{proof}[Proof of {#1}]}{\end{proof}}

%% \email{} command
\providecommand{\email}[1]{\href{mailto:#1}{\nolinkurl{#1}\xspace}}

%% Remarks and notes
\ifnum\withcolors=1
  \newcommand{\new}[1]{{\color{red} {#1}}} % new
  \newcommand{\newer}[1]{{\color{blue} {#1}}} % even newer
  \newcommand{\newest}[1]{{\color{orange} {#1}}} % even even newer
  \newcommand{\newerest}[1]{{\color{blue!10!black!40!green} {#1}}} % you get the idea.
  \newcommand{\ccolor}[1]{{\color{RubineRed}#1}} % Clement
\else
  \newcommand{\new}[1]{{{#1}}}
  \newcommand{\newer}[1]{{{#1}}}
  \newcommand{\newest}[1]{{{#1}}}
  \newcommand{\newerest}[1]{{{#1}}}
  \newcommand{\ccolor}[1]{{#1}}
\fi

\ifnum\withnotes=1
  \newcommand{\cnote}[1]{\par\ccolor{\textbf{C: }\sf #1}} % Clement
  \newcommand{\todonote}[2][]{\todo[size=\scriptsize,color=red!40,#1]{#2}}  
	\newcommand{\questionnote}[2][]{\todo[size=\scriptsize,color=blue!30]{#2}}
	\newcommand{\todonotedone}[2][]{\todo[size=\scriptsize,color=green!40]{$\checkmark$ #2}}
	\newcommand{\todonoteinline}[2][]{\todo[inline,size=\scriptsize,color=orange!40,#1]{#2}}  
  \newcommand{\marginnote}[1]{\todo[color=white,linecolor=black]{{#1}}}
\else
  \newcommand{\cnote}[1]{}
  \newcommand{\todonote}[2][]{\ignore{#2}}
	\newcommand{\questionnote}[2][]{\ignore{#2}}
	\newcommand{\todonotedone}[2][]{\ignore{#2}}
	\newcommand{\todonoteinline}[2][]{\ignore{#2}}
  \newcommand{\marginnote}[1]{\ignore{#1}}
\fi
\newcommand{\ignore}[1]{\leavevmode\unskip} % eat unnecessary spaces before
\newcommand{\cmargin}[1]{\questionnote{\ccolor{#1}}} % Clement

% Shortcuts
\newcommand{\eps}{\ensuremath{\varepsilon}\xspace}
\newcommand{\Algo}{\ensuremath{\mathcal{A}}\xspace} % Algorithm A
\newcommand{\Tester}{\ensuremath{\mathcal{T}}\xspace} % Testing algorithm T
\newcommand{\Learner}{\ensuremath{\mathcal{L}}\xspace} % Learning algorithm L
\newcommand{\property}{\ensuremath{\mathcal{P}}\xspace} % Property P
\newcommand{\class}{\ensuremath{\mathcal{C}}\xspace} % Class C
\newcommand{\eqdef}{\stackrel{\rm def}{=}}
\newcommand{\eqlaw}{\stackrel{\mathcal{L}}{=}}
\newcommand{\accept}{\textsf{ACCEPT}\xspace}
\newcommand{\fail}{\textsf{FAIL}\xspace}
\newcommand{\reject}{\textsf{REJECT}\xspace}
\newcommand{\opt}{{\textsc{opt}}\xspace}
\newcommand{\half}{\frac{1}{2}}
\newcommand{\domain}{\ensuremath{\Omega}\xspace} % Domain of a distribution (default notation)
\newcommand{\distribs}[1]{\Delta\!\left(#1\right)} % Domain of a distribution (default notation)
\newcommand{\yes}{{\sf{}yes}\xspace}
\newcommand{\no}{{\sf{}no}\xspace}
\newcommand{\dyes}{{\cal Y}}
\newcommand{\dno}{{\cal N}}

% Complexity
\newcommand{\littleO}[1]{{o\mleft( #1 \mright)}}
\newcommand{\bigO}[1]{{O\mleft( #1 \mright)}}
\newcommand{\bigOSmall}[1]{{O\big( #1 \big)}}
\newcommand{\bigTheta}[1]{{\Theta\mleft( #1 \mright)}}
\newcommand{\bigOmega}[1]{{\Omega\mleft( #1 \mright)}}
\newcommand{\bigOmegaSmall}[1]{{\Omega\big( #1 \big)}}
\newcommand{\tildeO}[1]{\tilde{O}\mleft( #1 \mright)}
\newcommand{\tildeTheta}[1]{\operatorname{\tilde{\Theta}}\mleft( #1 \mright)}
\newcommand{\tildeOmega}[1]{\operatorname{\tilde{\Omega}}\mleft( #1 \mright)}
\providecommand{\poly}{\operatorname*{poly}}

% Influence
\newcommand{\totinf}[1][f]{{\mathbf{Inf}[#1]}}
\newcommand{\infl}[2][f]{{\mathbf{Inf}_{#1}(#2)}}
\newcommand{\infldeg}[3][f]{{\mathbf{Inf}_{#1}^{#2}(#3)}}

% Sets and indicators
\newcommand{\setOfSuchThat}[2]{ \left\{\; #1 \;\colon\; #2\; \right\} } 			% sets such as "{ elems | condition }"
\newcommand{\indicSet}[1]{\mathds{1}_{#1}}                                              % indicator function
\newcommand{\indic}[1]{\indicSet{\left\{#1\right\}}}                                             % indicator function
\newcommand{\disjunion}{\amalg}%\coprod, \dotcup...

% Distance
\newcommand{\dtv}{\operatorname{d}_{\rm TV}}
\newcommand{\kl}{\operatorname{KL}}
\newcommand{\dhell}{\operatorname{d_{\rm{}H}}}
\newcommand{\hellinger}[2]{{\dhell\mleft({#1, #2}\mright)}}
\newcommand{\kldiv}[2]{{\kl\mleft({#1 \,\|\, #2}\mright)}}
\newcommand{\kolmogorov}[2]{{\operatorname{d_{\rm{}K}}\mleft({#1, #2}\mright)}}
\newcommand{\totalvardistrestr}[3][]{{\dtv^{#1}\mleft({#2, #3}\mright)}}
\newcommand{\totalvardist}[2]{\totalvardistrestr[]{#1}{#2}}
%\newcommand{\chisquarerestr}[3][]{{\operatorname{d}^{#1}_{\chi^2}\mleft({#2 \mid\mid #3}\mright)}}
\newcommand{\chisquare}[2]{{\chi^2\mleft({#1 \mid\mid #2}\mright)}}
\newcommand{\dist}[2]{\operatorname{dist}\mleft({#1, #2}\mright)}

% Restriction (functions, sequences, etc.)
\newcommand\restr[2]{{% we make the whole thing an ordinary symbol
  \left.\kern-\nulldelimiterspace % automatically resize the bar with \right
  #1 % the function
  \vphantom{\big|} % pretend it's a little taller at normal size
  \right|_{#2} % this is the delimiter
  }}

% Probability
\newcommand{\proba}{\Pr}
\newcommand{\probaOf}[1]{\proba\!\left[\, #1\, \right]}
\newcommand{\probaCond}[2]{\proba\!\left[\, #1 \;\middle\vert\; #2\, \right]}
\newcommand{\probaDistrOf}[2]{\proba_{#1}\left[\, #2\, \right]}

% Support of a distribution/function
\newcommand{\supp}[1]{\operatorname{supp}\!\left(#1\right)}

% Expectation & variance
\newcommand{\expect}[1]{\mathbb{E}\!\left[#1\right]}
\newcommand{\expectCond}[2]{\mathbb{E}\!\left[\, #1 \;\middle\vert\; #2\, \right]}
\newcommand{\shortexpect}{\mathbb{E}}
\newcommand{\var}{\operatorname{Var}}

% Distributions
\newcommand{\uniform}{\ensuremath{\mathcal{U}}}
\newcommand{\uniformOn}[1]{\ensuremath{\uniform\!\left( #1 \right) }}
\newcommand{\geom}[1]{\ensuremath{\operatorname{Geom}\!\left( #1 \right)}}
\newcommand{\bernoulli}[1]{\ensuremath{\operatorname{Bern}\!\left( #1 \right)}}
\newcommand{\bern}[2]{\ensuremath{\operatorname{Bern}^{#1}\!\left( #2 \right)}}
\newcommand{\binomial}[2]{\ensuremath{\operatorname{Bin}\!\left( #1, #2 \right)}}
\newcommand{\poisson}[1]{\ensuremath{\operatorname{Poisson}\!\left( #1 \right) }}
\newcommand{\gaussian}[2]{\ensuremath{ \mathcal{N}\!\left(#1,#2\right) }}
\newcommand{\gaussianpdf}[2]{\ensuremath{ g_{#1,#2}}}
\newcommand{\betadistr}[2]{\ensuremath{ \operatorname{Beta}\!\left( #1, #2 \right) }}

% Norms
\newcommand{\norm}[1]{\lVert#1{\rVert}}
\newcommand{\normone}[1]{{\norm{#1}}_1}
\newcommand{\normtwo}[1]{{\norm{#1}}_2}
\newcommand{\norminf}[1]{{\norm{#1}}_\infty}
\newcommand{\abs}[1]{\left\lvert #1 \right\rvert}
\newcommand{\dabs}[1]{\lvert #1 \rvert}
\newcommand{\dotprod}[2]{ \left\langle #1,\xspace #2 \right\rangle } 			% <a,b>
\newcommand{\ip}[2]{\dotprod{#1}{#2}} 			% shortcut

\newcommand{\vect}[1]{\mathbf{#1}} 			% shortcut

% Ceiling and floor
\newcommand{\clg}[1]{\left\lceil #1 \right\rceil}
\newcommand{\flr}[1]{\left\lfloor #1 \right\rfloor}

% Common sets
\newcommand{\R}{\ensuremath{\mathbb{R}}\xspace}
\newcommand{\C}{\ensuremath{\mathbb{C}}\xspace}
\newcommand{\Q}{\ensuremath{\mathbb{Q}}\xspace}
\newcommand{\Z}{\ensuremath{\mathbb{Z}}\xspace}
\newcommand{\N}{\ensuremath{\mathbb{N}}\xspace}
\newcommand{\cont}[1]{\ensuremath{\mathcal{C}^{#1}}}

% Oracles and variants
\newcommand{\ICOND}{{\sf INTCOND}\xspace}
\newcommand{\EVAL}{{\sf EVAL}\xspace}
\newcommand{\CDFEVAL}{{\sf CEVAL}\xspace}
\newcommand{\STAT}{{\sf STAT}\xspace}
\newcommand{\SAMP}{{\sf SAMP}\xspace}
\newcommand{\COND}{{\sf COND}\xspace}
\newcommand{\PCOND}{{\sf PAIRCOND}\xspace}
\newcommand{\ORACLE}{{\sf ORACLE}\xspace}

%% Terminology
\newcommand{\pdfsamp}{dual\xspace}
\newcommand{\cdfsamp}{cumulative dual\xspace}
\newcommand{\Pdfsamp}{\expandafter\capitalisewords\expandafter{\pdfsamp}}
\newcommand{\Cdfsamp}{\expandafter\capitalisewords\expandafter{\cdfsamp}}

% L_p norms
\newcommand{\lp}[1][1]{\ell_{#1}}

% Convolution
\DeclareMathOperator{\convolution}{\ast}

%% Terminology
\newcommand{\D}{\ensuremath{D}}
\newcommand{\distrD}{\ensuremath{\mathcal{D}}}
\newcommand{\birge}[2][\D]{\Phi_{#2}(#1)}
\newcommand{\iid}{i.i.d.\xspace}

% Sign
\DeclareMathOperator{\sign}{sgn}

%% Roman numerals
\makeatletter
\newcommand{\rom}[1]{\romannumeral #1}
\newcommand{\Rom}[1]{\expandafter\@slowromancap\romannumeral #1@}
\newcommand{\century}[2][th]{\Rom{#2}\textsuperscript{#1}}
\makeatother

% Hyperref and \autoref{} -- names
\renewcommand{\sectionautorefname}{Section} % To have "Section 5" instead of "section 5" with \autoref{}
\renewcommand{\chapterautorefname}{Chapter} % To have "Chapter 5" instead of "chapter 5" with \autoref{}
\renewcommand{\subsectionautorefname}{Section} % To have "Section 5" instead of "subsection 5" with \autoref{}
\renewcommand{\subsubsectionautorefname}{Section} % To have "Section 5" instead of "subsubsection 5" with \autoref{}
\def\algorithmautorefname{Algorithm}


%%%%%%%%%%%%%%%%%%%%%%%%%%%%%%%%%%%%%%%%%%%%%%%%%%%%%%%%%%%%%%%%%
% Add author and title info to PDF (and handles multiple authors)
%%%%%%%%%%%%%%%%%%%%%%%%%%%%%%%%%%%%%%%%%%%%%%%%%%%%%%%%%%%%%%%%%
\makeatletter
  \AtBeginDocument{
  \begingroup
  \toks@={}%
  \toksdef\toks@@=2 %
  \toks@@={}%
  \long\def\@ReturnFiFi#1#2\fi\fi{\fi\fi#1}%
  \def\scan@author#1#2 \and#3\@nil{%
  \ifx\\#3\\%
    \ifcase#1 %
      \toks@={#2}%
    \else
      \ifnum#1>1 %
        \toks@=\expandafter{%
          \the\expandafter\toks@\expandafter,\expandafter\space
          \the\toks@@
        }%
      \fi
      \toks@=\expandafter{\the\toks@\space and #2}%
    \fi
    \else
      \ifcase#1 %
        \toks@={#2}%
        \@ReturnFiFi{%
          \scan@author1#3\@nil
        }%
      \else
        \ifnum#1>1 %
          \toks@=\expandafter{%
            \the\expandafter\toks@\expandafter,\expandafter\space
            \the\toks@@
          }%
      \fi
      \toks@@={#2}%
      \@ReturnFiFi{%
        \scan@author2#3\@nil
      }%
    \fi
  \fi
  }%
  \expandafter\expandafter\expandafter\scan@author
  \expandafter\expandafter\expandafter0%
  \expandafter\@author\space\and\@nil
  \edef\x{\endgroup
  \noexpand\hypersetup{pdfauthor={\the\toks@}}%
  }%
  \x
  }
\makeatother

% Algorithm environment
\usepackage{algorithmicx,algpseudocode,  algorithm}


% compact lists, and handy shortcuts for items
\usepackage[shortlabels]{enumitem}
\setitemize{noitemsep,topsep=0pt,parsep=0pt,partopsep=0pt}


\def\maintitle{Assouad and Le Cam -- proving distribution learning and testing lower bounds}
\def\authorname{Cl\'ement L. Canonne}

\title{\maintitle}
\author{\authorname}
\date{May, 2014}

\makeatletter
  \hypersetup{
    pdftitle={\maintitle},
    pdfauthor={\authorname}, 
    pdfsubject={Probability Distribution Testing and Learning},
    pdfkeywords={property testing} {distribution testing} {sublinear algorithms} {lower bounds} {Assouad} {Le Cam}
  }
\makeatother

\begin{document}
\begin{flushleft}\sf\footnotesize
\makeatletter
\@date~- \today \hfill \@title
\makeatother
\end{flushleft}
\vspace{5mm}

In this (short) note, we focus on two techniques used to prove lower bounds for distribution \emph{learning} and \emph{testing}, respectively Assouad's lemma and Le Cam's method. (We do not cover here Fano's lemma, another and somewhat more general result than Assouad's -- the interested reader is referred to~\cite{Yu:97}.)\medskip

Hereafter, we let $(\domain,\mathcal{B})$ be a measurable space, and $\distribs{\domain}$ be the set of all probability distributions on it. Let $\totalvardist{\cdot}{\cdot}$ denote the total variation distance (the theorem would actually apply to any metric $d$ on $\distribs{\domain}$), and $\hellinger{\cdot}{\cdot}$ be the \emph{Hellinger distance}, defined as
\[
\hellinger{\D}{\D^\prime}\eqdef\frac{1}{2}\normtwo{\sqrt{\D}-\sqrt{\D^\prime}} = \frac{1}{2}\sqrt{\sum_{x\in \domain}\left(\sqrt{\D(x)}-\sqrt{\D^\prime(x)}\right)^2} = \sqrt{1 - \sum_{x\in \domain} \sqrt{\D(x)D^\prime(x)}}
\]
(the last two expressions holding when $\domain$ is countable).

%%%%%%%%%% Learning
\section{Learning Lower Bounds: Assouad's Lemma}

\begin{definition}[Minimax Risk]
Let $\class\subseteq\distribs{\domain}$ be a family of probability distributions, and $m\geq 1$. The \emph{minimax risk for $\class$ with $m$ samples} (with relation to the total variation distance) is defined as
  \begin{align}\label{eq:minimax:risk}
    R_m(\class) &\eqdef \inf_{A\in \Algo_m} \sup_{\D\in\class } \shortexpect_{s_1,\dots,s_m\sim \D}\!\left[ \totalvardist{ \D }{ \hat{\D}_A } \right] \\
    &= \inf_{A\in\Algo_m} \sup_{\D\in\class } \int_{\domain^m} \totalvardist{ \D }{ A(\vect{s}) }D^{\otimes m}(d\vect{s}) \notag{}
  \end{align}
  where $\Algo_m$ is the set of \new{(deterministic)}\todonote{Check this}{} learning algorithms $A$ which take $m$ samples and output a hypothesis distribution $\hat{\D}_A$.
\end{definition}
\noindent In other terms, $R_m(\class)$ is the minimum expected error of any $m$-sample learning algorithm $A$ when run on the worst possible target distribution (from $\class$) for it. It is immediate from the definition that for any $\mathcal{H}\subseteq \class$, one has $R_m(\class) \geq R_m(\mathcal{H})$.\medskip

To prove lower bounds on learning a family $\class$, a very common method is to come up with a (sub)family of distributions in which, as long as a learning algorithm does not take enough samples, there always exist two (far) distributions which still could have yielded indistinguishable ``transcripts''. In other terms, after running any learning algorithm $A$ on $m$ samples, an adversary can still exhibit two very different distributions (depending on $A$)\footnote{Note that this differs from the standard methodology for proving lower bounds for property testing, where two families of distributions (\textsf{yes} and \textsf{no}-instances) are defined beforehand, and a couple of distributions is ``committed to'' \emph{before} the algorithm gets to make its move.} that \emph{ought} to be distinguished, yet \emph{could not} possibly have been from only $m$ samples. This is formalized by the following theorem, due to Assouad:
\begin{theorem}[Assouad's Lemma~\cite{Assouad:83}]
Let $\class\subseteq\distribs{\domain}$ be a family of probability distributions. Suppose there exists a family of $\mathcal{H}\subseteq\class$ of $2^r$ distributions and constants $\alpha,\beta > 0$ such that, writing $\mathcal{H}=\{\D_z\}_{z\in\{0,1\}^r}$,
\begin{enumerate}[(i)]
  \item\label{item:assouad:condition:1} for all $x,y\in\{0,1\}^r$, the distance between $D_x$ and $D_y$ is at least proportional to the Hamming distance:
    \begin{equation}\label{eq:assouad:condition:1}
      \totalvardist{\D_x}{\D_y} \geq \alpha \normone{x-y}
    \end{equation}
  \item\label{item:assouad:condition:2} for all $x,y\in\{0,1\}^r$ with  $\normone{x-y}=1$, the squared Hellinger distance of $D_x,D_y$ is small:
    \begin{equation}\label{eq:assouad:condition:2}
      \hellinger{\D_x}{\D_y}^2 \leq \beta
    \end{equation}
    (or, equivalently, $-\ln(1-\operatorname{h}^2) \leq \ln\frac{1}{1-\beta}$)
\end{enumerate}
Then, for all $m \geq 1$,
    \begin{equation}\label{eq:assouad:conclusion}
        R_m(\mathcal{H}) \geq \frac{1}{4}\alpha r (1-\beta)^{2m} = \bigOmega{\alpha r e^{-\bigO{\beta m}} }.
    \end{equation}
In particular, to achieve error at most $\eps$, any learning algorithm for $\class$ must have sample complexity $\bigOmega{\frac{1}{\beta}\log\frac{\alpha r}{\eps}}$.
\end{theorem}
\begin{remark}[High-level idea]
Intuitively, every distribution in $\mathcal{H}$ is defined by making $r$ distinct ``choices''\footnote{E.g., by choosing, for each of $r$ intervals partitioning the support, whether the distribution \textsf{(a)} is uniform on the interval or \textsf{(b)} puts all its weight on the first half of the interval.}. With this interpretation, \autoref{item:assouad:condition:1} means that two distributions differing in many choices should be far (so that a learning algorithm has to ``figure out'' \emph{most} of the choices in order to achieve a small error), while \autoref{item:assouad:condition:2} requires that two distributions defined by almost the same choices be very close (so that a learning algorithm cannot distinguish them \emph{too easily}).
\end{remark}

\begin{remark}[Technical detail]
The quantity $1-\hellinger{p}{q}^2$ is known as the \emph{Hellinger affinity}; as the Hellinger distance satisfies
\begin{equation}\label{eq:dtv:hellinger}
1 - \sqrt{1-\totalvardist{p}{q}^2} \leq \hellinger{p}{q}^2 \leq \totalvardist{p}{q}
\end{equation}
it is sufficient for \eqref{eq:assouad:condition:2} to show that the (sometimes easier) condition holds:
\[
\totalvardist{\D_x}{\D_y} \leq \beta.
\]
Note that, with \eqref{eq:assouad:condition:1} this imposes that $\alpha \leq \beta$; while working with the Hellinger distance only requires $\alpha^2 \leq 2\beta-\beta^2$ (from \eqref{eq:dtv:hellinger} and \eqref{eq:assouad:condition:1}).
\end{remark}

\paragraph*{An example of application.} To prove a lower bound of $\bigOmega{\frac{\log n}{\eps^3}}$ for learning monotone distributions over $[n]$, Birg\'e~\cite{Birge:87} invokes Assouad's Lemma, defining a family $\mathcal{H}$ achieving parameters $r=\bigTheta{\frac{\log n}{\eps}}$, $\alpha=\bigTheta{\eps/r}$ and $\beta=\bigTheta{\eps^2/r}$. This example shows a very neat feature of Assouad's Lemma -- \emph{it enables us to get a dependence on $\eps$ in the lower bound.}

%%%%%%%%%% Testing
\section{Testing Lower Bounds: Le Cam's Method}

We now turn to another lower bound technique, better suited for proving lower bounds on property testing or parameter estimation -- i.e., where the quantity of interest is a functional of the unknown distribution, instead of the distribution itself. We begin with some terminology that will be useful in stating the main result of this section.
\begin{definition}
Let $\class\subseteq\distribs{\domain}$ be a family of probability distributions over \domain, and $m\geq 1$. The \emph{convex hull of $m$-product distributions from \class}, denoted $\operatorname{conv}_m(\class)$, it the set of probability distributions over $\domain^q$ defined as
\[
  \operatorname{conv}_m(\class) \eqdef \setOfSuchThat{ \sum_{k=1}^\ell \alpha_k \D_k^{\otimes m} }{  \ell \geq 1, \D_1,\dots,\D_\ell\in\class, \alpha_1,\dots,\alpha_\ell \geq 0, \sum_{k=1}^\ell \alpha_k =1 }.
\]
That is, $\operatorname{conv}_m(\class)$ is the set of mixtures of $m$-wise product distributions from $\class$. (Note that distributions in $\operatorname{conv}_m(\class)$ are not in general product distributions themselves.)
\end{definition}
\begin{definition}[Estimator]
Let $\class\subseteq\distribs{\domain}$ be a family of probability distributions over \domain, and $m\geq 1$. For any real-valued functional $\varphi\colon\class\to[0,1]$ (``scalar property''), we denote by $\mathcal{E}_m$ the set of \emph{estimators} for $\varphi$: that is, the set of \new{(deterministic)} algorithms $E$ taking $m\geq 1$ independent samples from a distribution $\D\in\class$ and outputting an estimate $\hat{\varphi}_E$ of $\varphi(\D)$.
\end{definition}

We state the following lemma for estimators taking value in $[0,1]$ endowed with the distance $\abs{\cdot}$, but it holds for more general metric spaces, and in particular for $([0,1], \normtwo{\cdot})$.
\begin{theorem}[Le Cam's Method~\cite{LeCam:73,LeCam:86,Yu:97}]\label{theo:lecam:method}
  Let $\class\subseteq\distribs{\domain}$ be a family of probability distributions over \domain, and let $\varphi\colon\class\to[0,1]$ be a scalar property. 
  Suppose there exists $\gamma \in [0,1]$, subsets $A_1,A_2\subseteq[0,1]$, and families $\mathcal{\D}_1,\mathcal{\D}_2\subseteq\class$ such that the following holds.
  \begin{enumerate}[(i)]
    \item\label{eq:lecam:condition:1} $A_1$ and $A_2$ are \emph{$\gamma$-separated}: $\abs{\alpha_1-\alpha_2} \geq \gamma$ for all $\alpha_1\in A_1, \alpha_2\in A_2$;
    \item\label{eq:lecam:condition:2} $\varphi(\mathcal{\D}_1) \subseteq A_1$ and $\varphi(\mathcal{\D}_2) \subseteq A_2$.
  \end{enumerate}
  Then, for all $m\geq 1$,
  \begin{equation}\label{eq:lecam:conclusion}
    \inf_{E\in \mathcal{E}_m} \sup_{\D\in\class } \shortexpect_{s_1,\dots,s_m\sim \D}\!\left[ \abs{\hat{\varphi}_E - \varphi(\D)} \right] 
    \geq \frac{\gamma}{2}\Big(1-\inf_{\substack{ p_1 \in \operatorname{conv}_m(\mathcal{\D}_1) \\ p_2 \in \operatorname{conv}_m(\mathcal{\D}_2) }} \totalvardist{p_1}{p_2}\Big).
  \end{equation}
\end{theorem}
One particular interest of this result is that the infimum is taken over the \emph{convex hull} of the $m$-fold product distributions from the families $\mathcal{\D}_1$ and $\mathcal{\D}_2$, and not over the $m$-fold distributions themselves. While this makes the computations much less straightforward (as a mixture of product distributions is not in general itself a product distribution, one can no longer rely on using the Hellinger distance as a proxy for total variation and leverage its nice properties with regard to product distributions), it also usually yields much tighter bounds -- as the infimum over the convex hull is often significantly smaller.

We now state an immediate corollary in terms of property testing, where a testing algorithm is said to \emph{fail} if it outputs \accept on a \no-instance or \reject on a \yes-instance. Note as usual that if the samples originate from a distribution which is neither a \yes nor \no-instance, then the any output is valid and the tester cannot fail.
\begin{corollary}\label{coro:lecam:pt}
  Fix $\eps\in(0,1)$, and a property $\property\subseteq\distribs{\domain}$. Let $\mathcal{\D}_1,\mathcal{\D}_2\subseteq\distribs{\domain}$ be families of respectively \yes- and \no-instances, i.e. such that $\mathcal{\D}_1\subseteq\property$, while any $\D\in\mathcal{\D}_2$ has $\totalvardist{\D}{\property} > \eps$. Then, for all $m\geq 1$,
  \begin{equation}\label{eq:lecam:pt:conclusion}
    \inf_{T\in \Tester_m} \sup_{\D\in\distribs{\domain} } \probaDistrOf{s_1,\dots,s_m\sim \D}{T(s_1,\dots,s_m) \text{ fails}}
    \geq \frac{1}{2}\Big(1-\inf_{\substack{ p_1 \in \operatorname{conv}_m(\mathcal{\D}_1) \\ p_2 \in \operatorname{conv}_m(\mathcal{\D}_2) }} \totalvardist{p_1}{p_2}\Big).
  \end{equation}
  where $\Tester_m$ is the set of \new{(deterministic)} testing algorithms $T$ with sample complexity $m$.
\end{corollary}
\noindent As any (possibly randomized) \textit{bona fide} testing algorithm can only fail with probability $1/3$, the above combined with Yao's Principle implies a lower bound of $\bigOmega{m}$ as soon as $m$ and $\mathcal{\D}_1,\mathcal{\D}_2$ satisfy
$
\inf_{p_1,p_2} \totalvardist{p_1}{p_2} < 1/3
$
in~\eqref{eq:lecam:pt:conclusion}.
\begin{proofof}{\autoref{coro:lecam:pt}}
We apply~\autoref{theo:lecam:method} with the following parameters: $A_1=\{0\}$, $A_2=\{1\}$, $\gamma=1$, and $\varphi\colon\D\in \class\mapsto \indicSet{\property}(\D)\in\{0,1\}$, where $\class=\property\cup\setOfSuchThat{ \D\in\distribs{\domain} }{ \totalvardist{\D}{\property} > \eps }$ is the set of valid instances.
\end{proofof}

\paragraph*{An example of application.} To prove a lower bound of $\bigOmega{\sqrt{n}/\eps^2}$ for testing uniformity over~$[n]$, Paninski~\cite{Paninski:08} defines the families $\mathcal{\D}_1=\property=\{\uniform_n\}$ and $\mathcal{\D}_2$ as the set of distributions $\D$ obtained by perturbing each disjoint pair of consecutive elements $(2i-1,2i)$ by either $(\frac{\eps}{n},-\frac{\eps}{n})$ or $(-\frac{\eps}{n},\frac{\eps}{n})$ (for a total of $2^{\frac{n}{2}}$ distinct distributions). He then analyzes the total variation distance between $\uniform_n^{\otimes m}$ and the uniform mixture
\[
  p \eqdef \frac{1}{2^{\frac{n}{2}}} \sum_{\D\in \mathcal{\D}_2} \D^{\otimes m}.
\]
By an approach similar as that of \cite[Section 14.4]{Pollard:2003}, Paninski shows that $\inf_{p_2 \in \operatorname{conv}_m(\mathcal{\D}_2) } \totalvardist{\uniform_n^{\otimes m}}{p_2} \leq \totalvardist{\uniform_n^{\otimes m}}{p} \leq \frac{1}{2}\sqrt{e^{m^2\eps^4/n} -1}$, which for $m \leq \frac{c\sqrt{n}}{\eps^2}$ is less than $1/3$ -- establishing the lower bound.

%%%%%%%%%% Bibliography
\begin{filecontents}{references1.bib}

@article {Assouad:83,
    author = {Assouad, Patrice},
     title = {Deux remarques sur l'estimation},
   journal = {Comptes Rendus des S\'eances de l'Acad\'emie des Sciences.
              S\'erie I. Math\'ematique},
    volume = {296},
      year = {1983},
    number = {23},
     pages = {1021--1024},
      issn = {0249-6291},
}

@article{Birge:87,
  ajournal = {The Annals of Statistics},
  author = "Birg\'e, Lucien",
  doi = "10.1214/aos/1176350488",
  journal = "The Annals of Statistics",
  month = "09",
  number = "3",
  pages = "995--1012",
  publisher = "The Institute of Mathematical Statistics",
  title = "Estimating a {D}ensity under {O}rder {R}estrictions: {N}onasymptotic {M}inimax {R}isk",
  url = "http://dx.doi.org/10.1214/aos/1176350488",
  volume = "15",
  year = "1987"
}

@incollection{Yu:97,
  year={1997},
  isbn={978-1-4612-7323-3},
  booktitle={Festschrift for Lucien Le Cam},
  editor={Pollard, David and Torgersen, Erik and Yang, Grace L.},
  doi={10.1007/978-1-4612-1880-7_29},
  title={{A}ssouad, {F}ano, and {L}e {C}am},
  url={http://dx.doi.org/10.1007/978-1-4612-1880-7_29},
  publisher={Springer New York},
  author={Yu, Bin},
  pages={423-435},
  language={English}
}

@article{LeCam:73,
  title={Convergence of estimates under dimensionality restrictions},
  author={Le Cam, Lucien},
  journal={The Annals of Statistics},
  pages={38--53},
  year={1973},
  volume = 1,
  issn = {0090-5364},
}

@book{LeCam:86,
    author = {Le Cam, Lucien},
     title = {Asymptotic methods in statistical decision theory},
    series = {Springer Series in Statistics},
 publisher = {Springer-Verlag, New York},
      year = {1986},
     pages = {xxvi+742},
      isbn = {0-387-96307-3},
       doi = {10.1007/978-1-4612-4946-7},
       url = {http://dx.doi.org/10.1007/978-1-4612-4946-7},
}

@article{Paninski:08,
  author    = {Paninski, Liam},
  title     = {A Coincidence-Based Test for Uniformity Given Very Sparsely
               Sampled Discrete Data},
  journal   = {IEEE Transactions on Information Theory},
  volume    = {54},
  number    = {10},
  year      = {2008},
  pages     = {4750-4755},
  ee        = {http://dx.doi.org/10.1109/TIT.2008.928987}
}

@misc{Pollard:2003,
  author = {Pollard, David},
  title = {Asymptopia},
  howpublished = {\url{http://www.stat.yale.edu/~pollard/Books/Asymptopia}},
  note = {Manuscript},
  year = 2003
}


\end{filecontents}
%%%%%%%%%%%%%%%%%%%%%%%%%%%%%%%%%%%%%%%%%%%%%%%%%%%%%%%%%%%%%%

\nocite{*}
\bibliographystyle{alpha}
\bibliography{references1}

\end{document}
