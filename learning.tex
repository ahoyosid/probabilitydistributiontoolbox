\documentclass[10pt]{article}
\def\withcolors{1}
\def\withnotes{1}
\def\withindex{0}
\usepackage[T1]{fontenc}
\usepackage[utf8]{inputenc}

%% Eye-candy
\usepackage{lmodern}
\usepackage{xspace}                                     % Smart spacing with \xspace
\usepackage[protrusion=true,expansion=true]{microtype}  % Improve font rendering

% Striking out text
\usepackage[normalem]{ulem}

%% Math
\usepackage{amsfonts,amsmath,amssymb, amsthm, mathtools}
\usepackage{thm-restate}
\usepackage{dsfont} % For the indicator symbol

% Algorithm environment
\usepackage{algorithmicx,algpseudocode,algorithm}

% Colors (with names)
\usepackage[usenames,dvipsnames,table]{xcolor}

% Quotes: \blockquote command
\usepackage{csquotes}

% Relative sizes for text
\usepackage{relsize}

% Bibliography
%\usepackage[numbers]{natbib}

% Required for the table of results
\usepackage{multirow}
\usepackage{chngpage} % allows for temporary adjustment of side margins

% For the commands such as \capitalisewords
\usepackage{mfirstuc}

% Graphics
\usepackage{tikz}
\usetikzlibrary{arrows}
\usetikzlibrary{calc,decorations.pathmorphing,patterns}

% For indexing
\ifnum\withindex=1
  \usepackage{makeidx}
  \usepackage{ifthen}
  \newcommand\indexed[2][]{\ifthenelse{\equal{#1}{}}{#2\index{#2}}{#2\index{#1}}}
  \makeindex %%%% Enable indexing
\fi
%%\usepackage{showidx} % To debug; does not play well with hyperref

% References and links
\usepackage[backref,colorlinks,citecolor=blue,bookmarks=true,linktocpage]{hyperref}
\usepackage{aliascnt}
\usepackage[numbered]{bookmark}

% Full pages
\usepackage{fullpage}

% Titling
\usepackage{titling}

% Compressed lists
\usepackage[shortlabels]{enumitem}
  \setitemize{noitemsep,topsep=3pt,parsep=2pt,partopsep=2pt} % Uncomment for compact item lists
  \setenumerate{itemsep=1pt,topsep=2pt,parsep=2pt,partopsep=2pt}
  \setdescription{itemsep=1pt}
  
% Package for todo notes.
\ifnum\withnotes=1
  \usepackage[colorinlistoftodos,textsize=scriptsize]{todonotes}
\fi

% Verbatim inputs and code
\usepackage{verbatim}

% Resizable parentheses that work (without the space between \left(#1\right)
\usepackage{mleftright} % \mleft( #1 \mright)

\makeatletter
\@ifundefined{theorem}{%
  % Theorems (each with its own style, all same counter). Cf. http://ftp.math.purdue.edu/mirrors/ctan.org/macros/latex/contrib/hyperref/doc/manual.pdf, p.17
  \theoremstyle{plain} %% Style
  	\newtheorem{theorem}{Theorem}%[section]
  	\newaliascnt{coro}{theorem}
  	  \newtheorem{corollary}[coro]{Corollary}
  	\aliascntresetthe{coro}
  	\newaliascnt{lem}{theorem}
  		\newtheorem{lemma}[lem]{Lemma}
  	\aliascntresetthe{lem}
  	\newaliascnt{clm}{theorem}
  		\newtheorem{claim}[clm]{Claim}
	\aliascntresetthe{clm}
	\newaliascnt{fact}{theorem}
 	 	\newtheorem{fact}[theorem]{Fact}
	\aliascntresetthe{fact}
  	\newtheorem*{unnumberedfact}{Fact}
  \newaliascnt{prop}{theorem}
  		\newtheorem{proposition}[prop]{Proposition}
	\aliascntresetthe{prop}
	\newaliascnt{conj}{theorem}
  		\newtheorem{conjecture}[conj]{Conjecture}
	\aliascntresetthe{conj}
  \theoremstyle{remark} %% Style
  	\newtheorem{remark}[theorem]{Remark}
  	\newtheorem{question}[theorem]{Question}
  	\newtheorem*{notation}{Notation}
 	 \newtheorem{example}[theorem]{Example}
  \theoremstyle{definition} %% Style
  	\newaliascnt{defn}{theorem}
 		 \newtheorem{definition}[defn]{Definition}
 	 \aliascntresetthe{defn}
}{}
\makeatother
\providecommand*{\lemautorefname}{Lemma} % For \autoref{} to know the name of lemmas
\providecommand*{\clmautorefname}{Claim}
\providecommand*{\propautorefname}{Proposition}
\providecommand*{\coroautorefname}{Corollary}
\providecommand*{\defnautorefname}{Definition}
\newenvironment{proofof}[1]{\begin{proof}[Proof of {#1}]}{\end{proof}}

%% \email{} command
\providecommand{\email}[1]{\href{mailto:#1}{\nolinkurl{#1}\xspace}}

%% Remarks and notes
\ifnum\withcolors=1
  \newcommand{\new}[1]{{\color{red} {#1}}} % new
  \newcommand{\newer}[1]{{\color{blue} {#1}}} % even newer
  \newcommand{\newest}[1]{{\color{orange} {#1}}} % even even newer
  \newcommand{\newerest}[1]{{\color{blue!10!black!40!green} {#1}}} % you get the idea.
  \newcommand{\ccolor}[1]{{\color{RubineRed}#1}} % Clement
\else
  \newcommand{\new}[1]{{{#1}}}
  \newcommand{\newer}[1]{{{#1}}}
  \newcommand{\newest}[1]{{{#1}}}
  \newcommand{\newerest}[1]{{{#1}}}
  \newcommand{\ccolor}[1]{{#1}}
\fi

\ifnum\withnotes=1
  \newcommand{\cnote}[1]{\par\ccolor{\textbf{C: }\sf #1}} % Clement
  \newcommand{\todonote}[2][]{\todo[size=\scriptsize,color=red!40,#1]{#2}}  
	\newcommand{\questionnote}[2][]{\todo[size=\scriptsize,color=blue!30]{#2}}
	\newcommand{\todonotedone}[2][]{\todo[size=\scriptsize,color=green!40]{$\checkmark$ #2}}
	\newcommand{\todonoteinline}[2][]{\todo[inline,size=\scriptsize,color=orange!40,#1]{#2}}  
  \newcommand{\marginnote}[1]{\todo[color=white,linecolor=black]{{#1}}}
\else
  \newcommand{\cnote}[1]{}
  \newcommand{\todonote}[2][]{\ignore{#2}}
	\newcommand{\questionnote}[2][]{\ignore{#2}}
	\newcommand{\todonotedone}[2][]{\ignore{#2}}
	\newcommand{\todonoteinline}[2][]{\ignore{#2}}
  \newcommand{\marginnote}[1]{\ignore{#1}}
\fi
\newcommand{\ignore}[1]{\leavevmode\unskip} % eat unnecessary spaces before
\newcommand{\cmargin}[1]{\questionnote{\ccolor{#1}}} % Clement

% Shortcuts
\newcommand{\eps}{\ensuremath{\varepsilon}\xspace}
\newcommand{\Algo}{\ensuremath{\mathcal{A}}\xspace} % Algorithm A
\newcommand{\Tester}{\ensuremath{\mathcal{T}}\xspace} % Testing algorithm T
\newcommand{\Learner}{\ensuremath{\mathcal{L}}\xspace} % Learning algorithm L
\newcommand{\property}{\ensuremath{\mathcal{P}}\xspace} % Property P
\newcommand{\class}{\ensuremath{\mathcal{C}}\xspace} % Class C
\newcommand{\eqdef}{\stackrel{\rm def}{=}}
\newcommand{\eqlaw}{\stackrel{\mathcal{L}}{=}}
\newcommand{\accept}{\textsf{ACCEPT}\xspace}
\newcommand{\fail}{\textsf{FAIL}\xspace}
\newcommand{\reject}{\textsf{REJECT}\xspace}
\newcommand{\opt}{{\textsc{opt}}\xspace}
\newcommand{\half}{\frac{1}{2}}
\newcommand{\domain}{\ensuremath{\Omega}\xspace} % Domain of a distribution (default notation)
\newcommand{\distribs}[1]{\Delta\!\left(#1\right)} % Domain of a distribution (default notation)
\newcommand{\yes}{{\sf{}yes}\xspace}
\newcommand{\no}{{\sf{}no}\xspace}
\newcommand{\dyes}{{\cal Y}}
\newcommand{\dno}{{\cal N}}

% Complexity
\newcommand{\littleO}[1]{{o\mleft( #1 \mright)}}
\newcommand{\bigO}[1]{{O\mleft( #1 \mright)}}
\newcommand{\bigOSmall}[1]{{O\big( #1 \big)}}
\newcommand{\bigTheta}[1]{{\Theta\mleft( #1 \mright)}}
\newcommand{\bigOmega}[1]{{\Omega\mleft( #1 \mright)}}
\newcommand{\bigOmegaSmall}[1]{{\Omega\big( #1 \big)}}
\newcommand{\tildeO}[1]{\tilde{O}\mleft( #1 \mright)}
\newcommand{\tildeTheta}[1]{\operatorname{\tilde{\Theta}}\mleft( #1 \mright)}
\newcommand{\tildeOmega}[1]{\operatorname{\tilde{\Omega}}\mleft( #1 \mright)}
\providecommand{\poly}{\operatorname*{poly}}

% Influence
\newcommand{\totinf}[1][f]{{\mathbf{Inf}[#1]}}
\newcommand{\infl}[2][f]{{\mathbf{Inf}_{#1}(#2)}}
\newcommand{\infldeg}[3][f]{{\mathbf{Inf}_{#1}^{#2}(#3)}}

% Sets and indicators
\newcommand{\setOfSuchThat}[2]{ \left\{\; #1 \;\colon\; #2\; \right\} } 			% sets such as "{ elems | condition }"
\newcommand{\indicSet}[1]{\mathds{1}_{#1}}                                              % indicator function
\newcommand{\indic}[1]{\indicSet{\left\{#1\right\}}}                                             % indicator function
\newcommand{\disjunion}{\amalg}%\coprod, \dotcup...

% Distance
\newcommand{\dtv}{\operatorname{d}_{\rm TV}}
\newcommand{\kl}{\operatorname{KL}}
\newcommand{\dhell}{\operatorname{d_{\rm{}H}}}
\newcommand{\hellinger}[2]{{\dhell\mleft({#1, #2}\mright)}}
\newcommand{\kldiv}[2]{{\kl\mleft({#1 \,\|\, #2}\mright)}}
\newcommand{\kolmogorov}[2]{{\operatorname{d_{\rm{}K}}\mleft({#1, #2}\mright)}}
\newcommand{\totalvardistrestr}[3][]{{\dtv^{#1}\mleft({#2, #3}\mright)}}
\newcommand{\totalvardist}[2]{\totalvardistrestr[]{#1}{#2}}
%\newcommand{\chisquarerestr}[3][]{{\operatorname{d}^{#1}_{\chi^2}\mleft({#2 \mid\mid #3}\mright)}}
\newcommand{\chisquare}[2]{{\chi^2\mleft({#1 \mid\mid #2}\mright)}}
\newcommand{\dist}[2]{\operatorname{dist}\mleft({#1, #2}\mright)}

% Restriction (functions, sequences, etc.)
\newcommand\restr[2]{{% we make the whole thing an ordinary symbol
  \left.\kern-\nulldelimiterspace % automatically resize the bar with \right
  #1 % the function
  \vphantom{\big|} % pretend it's a little taller at normal size
  \right|_{#2} % this is the delimiter
  }}

% Probability
\newcommand{\proba}{\Pr}
\newcommand{\probaOf}[1]{\proba\!\left[\, #1\, \right]}
\newcommand{\probaCond}[2]{\proba\!\left[\, #1 \;\middle\vert\; #2\, \right]}
\newcommand{\probaDistrOf}[2]{\proba_{#1}\left[\, #2\, \right]}

% Support of a distribution/function
\newcommand{\supp}[1]{\operatorname{supp}\!\left(#1\right)}

% Expectation & variance
\newcommand{\expect}[1]{\mathbb{E}\!\left[#1\right]}
\newcommand{\expectCond}[2]{\mathbb{E}\!\left[\, #1 \;\middle\vert\; #2\, \right]}
\newcommand{\shortexpect}{\mathbb{E}}
\newcommand{\var}{\operatorname{Var}}

% Distributions
\newcommand{\uniform}{\ensuremath{\mathcal{U}}}
\newcommand{\uniformOn}[1]{\ensuremath{\uniform\!\left( #1 \right) }}
\newcommand{\geom}[1]{\ensuremath{\operatorname{Geom}\!\left( #1 \right)}}
\newcommand{\bernoulli}[1]{\ensuremath{\operatorname{Bern}\!\left( #1 \right)}}
\newcommand{\bern}[2]{\ensuremath{\operatorname{Bern}^{#1}\!\left( #2 \right)}}
\newcommand{\binomial}[2]{\ensuremath{\operatorname{Bin}\!\left( #1, #2 \right)}}
\newcommand{\poisson}[1]{\ensuremath{\operatorname{Poisson}\!\left( #1 \right) }}
\newcommand{\gaussian}[2]{\ensuremath{ \mathcal{N}\!\left(#1,#2\right) }}
\newcommand{\gaussianpdf}[2]{\ensuremath{ g_{#1,#2}}}
\newcommand{\betadistr}[2]{\ensuremath{ \operatorname{Beta}\!\left( #1, #2 \right) }}

% Norms
\newcommand{\norm}[1]{\lVert#1{\rVert}}
\newcommand{\normone}[1]{{\norm{#1}}_1}
\newcommand{\normtwo}[1]{{\norm{#1}}_2}
\newcommand{\norminf}[1]{{\norm{#1}}_\infty}
\newcommand{\abs}[1]{\left\lvert #1 \right\rvert}
\newcommand{\dabs}[1]{\lvert #1 \rvert}
\newcommand{\dotprod}[2]{ \left\langle #1,\xspace #2 \right\rangle } 			% <a,b>
\newcommand{\ip}[2]{\dotprod{#1}{#2}} 			% shortcut

\newcommand{\vect}[1]{\mathbf{#1}} 			% shortcut

% Ceiling and floor
\newcommand{\clg}[1]{\left\lceil #1 \right\rceil}
\newcommand{\flr}[1]{\left\lfloor #1 \right\rfloor}

% Common sets
\newcommand{\R}{\ensuremath{\mathbb{R}}\xspace}
\newcommand{\C}{\ensuremath{\mathbb{C}}\xspace}
\newcommand{\Q}{\ensuremath{\mathbb{Q}}\xspace}
\newcommand{\Z}{\ensuremath{\mathbb{Z}}\xspace}
\newcommand{\N}{\ensuremath{\mathbb{N}}\xspace}
\newcommand{\cont}[1]{\ensuremath{\mathcal{C}^{#1}}}

% Oracles and variants
\newcommand{\ICOND}{{\sf INTCOND}\xspace}
\newcommand{\EVAL}{{\sf EVAL}\xspace}
\newcommand{\CDFEVAL}{{\sf CEVAL}\xspace}
\newcommand{\STAT}{{\sf STAT}\xspace}
\newcommand{\SAMP}{{\sf SAMP}\xspace}
\newcommand{\COND}{{\sf COND}\xspace}
\newcommand{\PCOND}{{\sf PAIRCOND}\xspace}
\newcommand{\ORACLE}{{\sf ORACLE}\xspace}

%% Terminology
\newcommand{\pdfsamp}{dual\xspace}
\newcommand{\cdfsamp}{cumulative dual\xspace}
\newcommand{\Pdfsamp}{\expandafter\capitalisewords\expandafter{\pdfsamp}}
\newcommand{\Cdfsamp}{\expandafter\capitalisewords\expandafter{\cdfsamp}}

% L_p norms
\newcommand{\lp}[1][1]{\ell_{#1}}

% Convolution
\DeclareMathOperator{\convolution}{\ast}

%% Terminology
\newcommand{\D}{\ensuremath{D}}
\newcommand{\distrD}{\ensuremath{\mathcal{D}}}
\newcommand{\birge}[2][\D]{\Phi_{#2}(#1)}
\newcommand{\iid}{i.i.d.\xspace}

% Sign
\DeclareMathOperator{\sign}{sgn}

%% Roman numerals
\makeatletter
\newcommand{\rom}[1]{\romannumeral #1}
\newcommand{\Rom}[1]{\expandafter\@slowromancap\romannumeral #1@}
\newcommand{\century}[2][th]{\Rom{#2}\textsuperscript{#1}}
\makeatother

% Hyperref and \autoref{} -- names
\renewcommand{\sectionautorefname}{Section} % To have "Section 5" instead of "section 5" with \autoref{}
\renewcommand{\chapterautorefname}{Chapter} % To have "Chapter 5" instead of "chapter 5" with \autoref{}
\renewcommand{\subsectionautorefname}{Section} % To have "Section 5" instead of "subsection 5" with \autoref{}
\renewcommand{\subsubsectionautorefname}{Section} % To have "Section 5" instead of "subsubsection 5" with \autoref{}
\def\algorithmautorefname{Algorithm}


%%%%%%%%%%%%%%%%%%%%%%%%%%%%%%%%%%%%%%%%%%%%%%%%%%%%%%%%%%%%%%%%%
% Add author and title info to PDF (and handles multiple authors)
%%%%%%%%%%%%%%%%%%%%%%%%%%%%%%%%%%%%%%%%%%%%%%%%%%%%%%%%%%%%%%%%%
\makeatletter
  \AtBeginDocument{
  \begingroup
  \toks@={}%
  \toksdef\toks@@=2 %
  \toks@@={}%
  \long\def\@ReturnFiFi#1#2\fi\fi{\fi\fi#1}%
  \def\scan@author#1#2 \and#3\@nil{%
  \ifx\\#3\\%
    \ifcase#1 %
      \toks@={#2}%
    \else
      \ifnum#1>1 %
        \toks@=\expandafter{%
          \the\expandafter\toks@\expandafter,\expandafter\space
          \the\toks@@
        }%
      \fi
      \toks@=\expandafter{\the\toks@\space and #2}%
    \fi
    \else
      \ifcase#1 %
        \toks@={#2}%
        \@ReturnFiFi{%
          \scan@author1#3\@nil
        }%
      \else
        \ifnum#1>1 %
          \toks@=\expandafter{%
            \the\expandafter\toks@\expandafter,\expandafter\space
            \the\toks@@
          }%
      \fi
      \toks@@={#2}%
      \@ReturnFiFi{%
        \scan@author2#3\@nil
      }%
    \fi
  \fi
  }%
  \expandafter\expandafter\expandafter\scan@author
  \expandafter\expandafter\expandafter0%
  \expandafter\@author\space\and\@nil
  \edef\x{\endgroup
  \noexpand\hypersetup{pdfauthor={\the\toks@}}%
  }%
  \x
  }
\makeatother


\newcommand{\dst}{\varepsilon}
\newcommand{\ab}{k}
\newcommand{\ns}{n}

\usepackage{filecontents}

\title{A short note on learning discrete distributions}
\date{February, 2020}

\begin{document}
\begin{flushleft}\sf\footnotesize
\makeatletter
\@date~- \today \hfill \@title
\makeatother
\end{flushleft}
\vspace{5mm}

The goal of this short note is to provide simple proofs for the ``folklore facts'' on the sample complexity of learning a discrete probability distribution over a known domain of size $\ab$ to total variation distance $\dst$, with error probability $\delta$. Thanks to \href{http://www.gautamkamath.com/}{Gautam Kamath} and \href{http://www.mit.edu/~jswright/}{John Wright} for suggesting ``someone should write this up as a note,'' and to \href{https://people.eecs.berkeley.edu/~jiantao/}{Jiantao Jiao} for discussions about the Hellinger case.\bigskip

For a given distance measure $\operatorname{d}$, we write $\Phi(\operatorname{d},\ab,\dst,\delta)$ for the sample complexity of learning discrete distributions over a known domain of size $\ab$, to accuracy $\dst>0$, with error probability $\delta\in(0,1]$. As usual, asymptotics will be taken with regard to $\ab$ going to infinity, $\dst$ going to $0$, and $\delta$ going to $0$, in that order.

\noindent Without loss of generality, we hereafter assume the domain is the set $[\ab]\eqdef \{1,\dots,\ab\}$.
\section{Total variation distance}

Recall that $\totalvardist{p}{q} = \sup_{S\subseteq [\ab]} (p(S)-q(S)) = \frac{1}{2}\normone{p-q}\in[0,1]$ for any $p,q\in\distribs{[\ab]}$. 
\begin{theorem}\label{theo:learning:tv}
  $\Phi(\dtv,\ab,\dst,\delta) = \bigTheta{\frac{\ab+\log(1/\delta)}{\dst^2}}$.
\end{theorem}
\noindent We focus here on the upper bound. The lower bound can be proven, e.g., via Assouad's lemma (for the $\ab/\dst^2$ term), and from the hardness of estimating the bias of a coin ($\ab=2$) with high probability (for the $\log(1/\delta)/\dst^2$ term).
\begin{proof}[First proof]
Consider the empirical distribution $\tilde{p}$ obtained by drawing $\ns$ independent samples $s_1,\dots,s_\ns$ from the underlying distribution $p\in\distribs{[\ab]}$:
\begin{equation}\label{def:empirical}
\tilde{p}(i) = \frac{1}{\ns} \sum_{j=1}^\ns \indic{s_j=i}, \qquad i\in [\ab]
\end{equation}
\begin{itemize}
  \item First, we bound the \emph{expected} total variation distance between $\tilde{p}$ and $p$, by using $\lp[2]$ distance as a proxy:
\[
    \expect{ \totalvardist{p}{\tilde{p}} }
    =\frac{1}{2}\expect{ \normone{p-\tilde{p}}} 
    =\frac{1}{2}\sum_{i=1}^\ab\expect{ \abs{p(i)-\tilde{p}(i)}}
    \leq\frac{1}{2}\sum_{i=1}^\ab\sqrt{\expect{ (p(i)-\tilde{p}(i))^2} }
\]
the last inequality by Jensen. But since, for every $i\in[\ab]$, $\ns\tilde{p}(i)$ follows a $\binomial{\ns}{p(i)}$ distribution, we have
$\expect{ (p(i)-\tilde{p}(i))^2} = \frac{1}{\ns^2}\var[\ns\tilde{p}(i)] = \frac{1}{\ns}p(i)(1-p(i))$, from which
\[
    \expect{ \totalvardist{p}{\tilde{p}} } \leq\frac{1}{2\sqrt{\ns}}\sum_{i=1}^\ab\sqrt{p(i)} \leq \frac{1}{2}\sqrt{\frac{\ab}{\ns}}
\]
the last inequality this time by Cauchy--Schwarz. Therefore, for $\ns\geq \frac{\ab}{\dst^2}$ we have $\expect{ \totalvardist{p}{\tilde{p}} }\leq \frac{\dst}{2}$.

  \item Next, to convert this expected result to a \emph{high probability} guarantee, we apply McDiarmid's inequality to the random variable $f(s_1,\dots,s_\ns) \eqdef \totalvardist{p}{\tilde{p}}$, noting that changing any single sample cannot change its value by more than $c\eqdef 1/\ns$:
\[
    \probaOf{ \abs{f(s_1,\dots,s_\ns) - \expect{f(s_1,\dots,s_\ns)}} \geq \frac{\dst}{2} } \leq 2e^{-\frac{2\left(\frac{\dst}{2}\right)^2}{\ns c^2}} = 2e^{-\frac{1}{2}\ns\dst^2}
\]
and therefore as long as $\ns\geq \frac{2}{\dst^2}\ln\frac{2}{\delta}$, we have $\abs{f(s_1,\dots,s_\ns) - \expect{f(s_1,\dots,s_\ns)}} \leq \frac{\dst}{2}$ with probability at least $1-\delta$. 
\end{itemize}
Putting it all together, we obtain that $\totalvardist{p}{\tilde{p}} \leq \dst$ with probability at least $1-\delta$, as long as $\ns\geq \max\left( \frac{\ab}{\dst^2},\frac{2}{\dst^2}\ln\frac{2}{\delta} \right)$.
\end{proof}

\begin{proof}[Second proof -- the ``fun'' one]
Again, we will analyze the behavior of the empirical distribution $\tilde{p}$ over $\ns$ i.i.d. samples from the unknown $p$ (cf.~\eqref{def:empirical}) -- because it is simple, efficiently computable, and \emph{it works}.  Recalling the definition of total variation distance, note that $\totalvardist{p}{\tilde{p}} > \dst$ literally means there exists a subset $S\subseteq [\ab]$ such that $\tilde{p}(S) > p(S) + \dst$. There are $2^\ab$ such subsets, so\dots{} let us do a union bound.

Fix any $S\subseteq[\ab]$. We have
\[
\tilde{p}(S) = \tilde{p}(i) \operatorname*{=}^{\eqref{def:empirical}} \frac{1}{\ns} \sum_{i\in S} \sum_{j=1}^\ns \indic{s_j=i}
\]
and so, letting $X_j \eqdef \sum_{i\in S}\indic{s_j=i}$ for $j\in [\ns]$, we have
$
\tilde{p}(S) = \frac{1}{\ns}\sum_{j=1}^\ns X_j
$ where the $X_j$'s are i.i.d. Bernoulli random variable with parameter $p(S)$. Here comes the Chernoff bound (actually, Hoeffding, the \emph{other} Chernoff):
\[
    \probaOf{ \tilde{p}(S) > p(S) + \dst } = \probaOf{ \frac{1}{\ns}\sum_{j=1}^\ns X_j > \expect{\frac{1}{\ns}\sum_{j=1}^\ns X_j} + \dst } \leq e^{-2\dst^2 \ns}
\]
and therefore $\probaOf{ \tilde{p}(S) > p(S) + \dst } \leq \frac{\delta}{2^\ab}$ for any $\ns\geq \frac{\ab\ln 2+\log(1/\delta)}{2\dst^2}$. A union bound over these $2^\ab$ possible sets $S$ concludes the proof:
\[
    \probaOf{ \exists S\subseteq [\ab] \text{ s.t. }\tilde{p}(S) > p(S) + \dst } \leq 2^\ab\cdot \frac{\delta}{2^\ab} = \delta
\]
and we are done. \emph{Badda bing badda boom}, as someone\footnote{John Wright.} would say.
\end{proof}

\section{Hellinger distance}

Recall that $\hellinger{p}{q} = \frac{1}{\sqrt{2}}\sqrt{\sum_{i=1}^\ab (\sqrt{p(i)}-\sqrt{q(i)})^2} = \frac{1}{\sqrt{2}}\normtwo{p-q}\in[0,1]$ for any $p,q\in\distribs{[\ab]}$. The Hellinger distance has many nice properties: it is well-suited to manipulating product distributions, its square is subadditive, and is always within a quadratic factor of the total variation distance; see, e.g.,~\cite[Appendix~C.2]{Canonne:15}.

\begin{theorem}\label{theo:learning:hellinger}
  $\Phi(\dhell,\ab,\dst,\delta) = \bigTheta{\frac{\ab+\log(1/\delta)}{\dst^2}}$.
\end{theorem}

This theorem is ``highly non-trivial'' to establish, however; for the sake of exposition, we will show increasingly stronger bounds, starting with the easiest to establish.
\begin{proposition}[Easy bound]\label{theo:learning:hellinger:easy}
  $\Phi(\dhell,\ab,\dst,\delta) = \bigO{\frac{\ab+\log(1/\delta)}{\dst^4}}$, and $\Phi(\dhell,\ab,\dst,\delta) = \bigOmega{\frac{\ab+\log(1/\delta)}{\dst^2}}$.
\end{proposition}
\begin{proof}
    This is immediate from~\autoref{theo:learning:tv}, recalling that $\frac{1}{2}\dtv^2\leq \dhell^2\leq \dtv$.
\end{proof}
\begin{proposition}[More involved bound]\label{theo:learning:hellinger:intermediate}
  $\Phi(\dhell,\ab,\dst,\delta) = \bigO{\frac{\ab}{\dst^2}+\frac{\log(1/\delta)}{\dst^4}}$.
\end{proposition}
\begin{proof}
    As for total variation distance, we consider the empirical distribution $\widehat{p}$ (cf.~\eqref{def:empirical}) obtained by drawing $\ns$ independent samples $s_1,\dots,s_\ns$ from $p\in\distribs{[\ab]}$.
    
    \begin{itemize}
      \item First, we bound the \emph{expected} squared Hellinger distance between $\widehat{p}$ and $p$: using the simple fact that
      $\hellinger{p}{q}^2 = 1-\sum_{i=1}^\ab \sqrt{p(i)q(i)}$ for any $p,q\in\distribs{[\ab]}$,
      \[
          \expect{ \hellinger{p}{\widehat{p}}^2 } = 1-\sum_{i=1}^\ab \sqrt{p(i)}\cdot \expect{\sqrt{\widehat{p}(i)}}\,.
      \]
      Now we would like to handle the square root inside the expectation, and \emph{of course} Jensen's inequality is in the wrong direction. However, for every nonnegative r.v. $X$ with positive expectation, letting $Y\eqdef X/\expect{X}$, we have that
      \begin{align*}
          \expect{\sqrt{X}} 
          &= \sqrt{\expect{X}}\cdot\expect{\sqrt{Y}}
          = \sqrt{\expect{X}}\cdot\expect{\sqrt{1+(Y-\expect{Y})})} \\
          &\geq \sqrt{\expect{X}} \mleft( 1+ \frac{1}{2}\expect{Y-\expect{Y}} - \frac{1}{6} \expect{(Y-\expect{Y})^2} \mright)
          = \sqrt{\expect{X}}\mleft(1-\frac{\var X}{6\expect{X}^2}\mright)
      \end{align*}
      where we used the inequality $\sqrt{1+x} \geq 1+\frac{x}{2}-\frac{x^2}{6}$, which holds for $x\geq 0$.\footnote{And is inspired by the Tayor expansion $\sqrt{1+x} = 1+\frac{x}{2} - \frac{x^2}{8} +o(x^2)$: there is \emph{some} intuition for it.}{} Since, for every $i\in[\ab]$, $\ns\widehat{p}(i)$ follows a $\binomial{\ns}{p(i)}$ distribution, we get
      \[
          \expect{ \hellinger{p}{\widehat{p}}^2 } \leq 1-\frac{1}{\sqrt{\ns}}\sum_{i=1}^\ab \sqrt{p(i)}\cdot\sqrt{\ns p(i)} \mleft(1-\frac{\ns p(i)(1-\ns p(i))}{6\ns^2 p(i)^2}\mright)
          \leq 1 - \sum_{i=1}^\ab p(i) \mleft(1-\frac{1}{6\ns p(i)}\mright) = \frac{\ab}{6\ns}\,.
      \]
      Therefore, for $\ns\geq \frac{\ab}{3\dst^2}$, we have $\expect{ \hellinger{p}{\widehat{p}}^2 }\leq \frac{\dst^2}{2}$.
      \item Next, to convert this expected result to a high probability guarantee, we \emph{would like} to apply McDiarmid's inequality to the random variable $f(s_1,\dots,s_\ns) \eqdef \hellinger{p}{\widehat{p}}^2$ as in the (first) proof of~\autoref{theo:learning:tv}; unfortunately, changing a sample can change the value by up to $c \approx 1/\sqrt{\ns}$, and McDiarmid will yield
only a vacuous bound.\footnote{Try it: it's a real bummer.}{} Instead, we will use a stronger, more involved concentration inequality:
      \begin{theorem}[{\cite[Theorem~8.6]{BLM:13}}]\label{theo:stronger:mcdiarmid}
          Let $f\colon \mathcal{X}^\ns \to \R$ be a measurable function, and let $X_1,\dots,X_\ns$ be independent random variables taking values in $\mathcal{X}$. Define $Z\eqdef f(X_1,\dots,X_\ns)$. Assume that there exist measurable functions $c_i\colon \mathcal{X}^\ns \to [0,\infty)$ such that, for all $x,y\in\mathcal{X}^\ns$,
          \[
              f(y) - f(x) \leq \sum_{i=1}^\ns c_i(x) \indic{x_i \neq y_i}\,.
          \]
          Then, setting $v \eqdef \shortexpect \sum_{i=1}^\ns c_i(x)^2$ and $v_\infty \eqdef \sup_{x\in\mathcal{X}^\ns} \sum_{i=1}^\ns c_i(x)^2$, we have, for all $t>0$,
          \[
              \probaOf{ Z \geq \expect{Z} + t } \leq e^{-\frac{t^2}{2v}}\,\qquad \probaOf{ Z \leq \expect{Z} - t } \leq e^{-\frac{t^2}{2v_\infty}}\,.
          \]
      \end{theorem}
      For our $f$ above, we have, for two any different $x, y \in [\ab]^\ns$ , that
      \begin{align*}
        f(y)-f(x) 
        &= \frac{1}{\sqrt{\ns}} \sum_{i=1}^\ab \sqrt{p(i)} \mleft( \sqrt{\sum_{j=1}^\ns \indic{x_j=i}} - \sqrt{\sum_{j=1}^\ns \indic{y_j=i}} \mright) \\
        &= \frac{1}{\sqrt{\ns}} \sum_{i=1}^\ab \sqrt{p(i)} \frac{\sum_{j=1}^\ns (\indic{x_j=i}-\indic{y_j=i})}{ \sqrt{\sum_{j=1}^\ns \indic{x_j=i}} + \sqrt{\sum_{j=1}^\ns \indic{y_j=i}} } \\
        &\leq \frac{1}{\sqrt{\ns}} \sum_{i=1}^\ab \sqrt{p(i)} \frac{\sum_{j=1}^\ns \indic{x_j=i}\indic{y_j\neq x_j}}{ \sqrt{\sum_{j=1}^\ns \indic{x_j=i}} }
        = \sum_{j=1}^\ns \underbrace{\sqrt{ \frac{ p_{x_j} }{ \ns\sum_{\ell=1}^\ns \indic{x_\ell=x_j} } }}_{c_j(x)} \cdot \indic{x_j\neq y_j}\,.
      \end{align*}
      In view of~\autoref{theo:stronger:mcdiarmid}, we then must evaluate
      \[
          v \eqdef \sum_{j=1}^\ns \expect{c_j(X)^2} = \frac{1}{\ns} \sum_{j=1}^\ns  \sum_{i=1}^\ab p(i)^2 \cdot \expect{ \frac{1}{1+ \sum_{\ell\neq j} \indic{X_\ell=i} } }
      \]
      where that last expectation is over $(x_\ell)_{\ell\neq j}$ drawn from $p^{\otimes(\ns-1)}$. Since $\sum_{\ell\neq j} \indic{X_\ell=i}$ is Binomially distributed with parameters $\ns-1$ and $p(i)$, we can use the simple fact that, for $N\sim\binomial{r}{\rho}$,
      \[
          \expect{\frac{1}{N+1}} = \frac{1-(1-\rho)^{r+1}}{\rho(r+1)} \leq \frac{1}{\rho(r+1)}
      \]
      to conclude that $v \leq \frac{1}{\ns^2} \sum_{j=1}^\ns \sum_{i=1}^\ab p(i)  = \frac{1}{\ns}$. By~\autoref{theo:stronger:mcdiarmid}, we obtain
      \[
          \probaOf{ \abs{f(s_1,\dots,s_\ns)-\expect{f(s_1,\dots,s_\ns)}} \geq \frac{\dst^2}{2} } \leq e^{-\frac{1}{8}\ns\dst^4}
      \]
      and therefore, as long as $\ns \geq \frac{8}{\dst^4}\ln\frac{1}{\delta}$, we have $\abs{f(s_1,\dots,s_\ns)-\expect{f(s_1,\dots,s_\ns)}} \leq \frac{\dst^2}{2}$ with probability at least $1-\delta$.
    \end{itemize}
    Putting it all together, we obtain that $\hellinger{p}{\widehat{p}}^2 \leq \dst^2$ with probability at least $1-\delta$, as long as $\ns \geq \max\mleft( \frac{\ab}{3\dst^2}, \frac{8}{\dst^4}\ln\frac{1}{\delta}\mright)$.
\end{proof}
\noindent We finally get to the final, optimal bound:
\begin{proofof}{\autoref{theo:learning:hellinger}}
We will rely on a recent~--~and quite involved~--~result due to Agrawal [Agr19], analyzing the concentration of the empirical distribution $\widehat{p}$ in terms of its Kullback--Leibler (KL) divergence with regard to the true $p$,
\[
    \kldiv{\widehat{p}}{p} = \sum_{i=1}^\ab \widehat{p}(i) \ln \frac{\widehat{p}(i)}{p(i)} \in[0,\infty]\,.
\]
Observing that $\hellinger{p}{q}^2 \leq \frac{1}{2} \kldiv{p}{q}$ for any distributions $p, q$, the aforementioned result is actually stronger than what we need:
\begin{theorem}[{\cite[Theorem~1.2]{Agrawal:19}}]
  Suppose $\ns \geq \frac{\ab-1}{\alpha}$. Then
  \[
      \probaOf{ \kldiv{\widehat{p}}{p} \geq \alpha } \leq e^{-\ns\alpha}\mleft(\frac{e\alpha\ns}{\ab-1}\mright)^{\ab-1}\,.
  \]
\end{theorem}
\noindent In view of the above relation between Hellinger and KL, we will apply this convergence result with $\alpha \eqdef 2\dst^2$, obtaining
  \[
      \probaOf{ \hellinger{\widehat{p}}{p} \geq \dst } \leq e^{-2\ns\dst^2 + (\ab-1)\ln\frac{2e\ns\dst^2}{\ab-1}}\,.
  \]
  \begin{fact}
      For $\ns \geq \frac{15}{2e}\frac{\ab}{\dst^2}$, we have $(\ab-1)\ln\frac{2e\ns\dst^2}{\ab-1}\leq \ns\dst^2$. 
  \end{fact}
  \begin{proof}
  The conclusion is equivalent to $2e\cdot\ln\frac{2e\ns\dst^2}{\ab-1}\leq \frac{2e\ns\dst^2}{\ab-1}$, and the conclusion follows from the fact that $x\geq 2e \ln x$ for $x\geq 15$.
  \end{proof}
  \noindent This fact implies that, for $\ns \geq \frac{15\ab}{2\dst^2}$, $\probaOf{ \hellinger{\widehat{p}}{p} \geq \dst } \leq e^{-\ns\dst^2}$. Overall, we obtain that $\hellinger{p}{\widehat{p}} \leq \dst$ with probability at least $1-\delta$ as long as $\ns \geq \max\mleft( \frac{15\ab}{2e\dst^2}, \frac{1}{\dst^2}\ln\frac{1}{\delta}\mright)$, as desired.
\end{proofof}

\section{$\chi^2$ and Kullback–-Leibler divergences}
To conclude, some remarks on Kullback--Leibler (KL) and chi-squared ($\chi^2$) divergences. Recall their definition, for $p,q\in\distribs{[\ab]}$,
\[
    \kldiv{p}{q} = \sum_{i=1}^\ab p(i) \ln \frac{p(i)}{q(i)}\,,\qquad \chisquare{p}{q} = \sum_{i=1}^\ab \frac{(p(i)-q(i))^2}{q(i)}
\]
both taking values in $[0,\infty]$; as well as the chain of (easily checked) inequalities
\[
    2\totalvardist{p}{q}^2 \leq \kldiv{p}{q} %\leq \ln(1+\chisquare{p}{q}) 
    \leq \chisquare{p}{q}\,,
\]
where the first one is Pinsker's. Importantly, KL and $\chi^2$ divergences are unbounded and asymmetric, so
the order of p and q matters \emph{a lot}: for instance, it is easy to show that, without strong assumptions on the
unknown distribution $p\in\distribs{[\ab]}$, the empirical estimator $\widehat{p}$ cannot achieve $\kldiv{p}{\widehat{p}} < \infty$ (resp., $\chisquare{p}{\widehat{p}} < \infty$) with any finite number of samples.\footnote{You can verify this: intuitively, the issue boils down to having to non-trivially learn even the elements of the support of $p$ that have arbitrarily small probability.}{} So, that's uplifting. (On the other hand, \emph{other} estimators than the empirical one, e.g., add-constant estimators, do provide good learning guarantees for those distance measures: see for instance~\cite{Kamath:15}).\smallskip

We are going to focus here on getting $\kldiv{\widehat{p}}{p}$ and $\chisquare{\widehat{p}}{p}$ down to $\dst$. Of course, in view of the inequalities above, the latter is at least as hard as the former, and a lower bound on both follows from that on $\dtv$: $\bigOmega{(\ab+\log(1/\delta))/\dst^2}$. And, behold! The result of Agrawal~\cite{Agrawal:19} used in the proof of~\autoref{theo:learning:hellinger} does provide the optimal upper bound on learning in KL divergence~--~and it is achieved by the usual suspect, the empirical estimator:
\begin{theorem}\label{theo:learning:kl}
  $\Phi(\kl,\ab,\dst,\delta) = \bigTheta{\frac{\ab+\log(1/\delta)}{\dst}}$, where by $\kl$ we refer to minimizing $\kldiv{\widehat{p}}{p}$.
\end{theorem}
\noindent The optimal sample complexity of learning in $\chi^2$ as a function of $\ab,\dst,\delta$, however, remains open.
%%%%%%%%%% Bibliography
\begin{filecontents}{references-learning.bib}
@article{Agrawal:19,
  author    = {Rohit Agrawal},
  title     = {Multinomial Concentration in Relative Entropy at the Ratio of Alphabet and Sample Sizes},
  journal   = {CoRR},
  volume    = {abs/1904.02291},
  year      = {2019}
}

@book{BLM:13,
  title={Concentration inequalities: A nonasymptotic theory of independence},
  author={Boucheron, St{\'e}phane and Lugosi, G{\'a}bor and Massart, Pascal},
  year={2013},
  publisher={Oxford University Press}
}


@article{Canonne:15,
  author    = {Cl{\'{e}}ment L. Canonne},
  title     = {{A Survey on Distribution Testing: Your Data is Big. But is it Blue?}},
  journal   = {Electronic Colloquium on Computational Complexity {(ECCC)}},
  volume    = {22},
  pages     = {63},
  year      = {2015}
}



@inproceedings{Kamath:15,
  title = 	 {On Learning Distributions from their Samples},
  author = 	 {Sudeep Kamath and Alon Orlitsky and Dheeraj Pichapati and Ananda Theertha Suresh},
  booktitle = 	 {Proceedings of The 28th Conference on Learning Theory},
  pages = 	 {1066--1100},
  year = 	 {2015},
  volume = 	 {40},
  series = 	 {Proceedings of Machine Learning Research},
  publisher = 	 {PMLR}
}

\end{filecontents}
%%%%%%%%%%%%%%%%%%%%%%%%%%%%%%%%%%%%%%%%%%%%%%%%%%%%%%%%%%%%%%
\bibliographystyle{alpha}
\bibliography{references-learning}
\end{document}
