\documentclass[10pt]{article}
\def\withcolors{1}
\def\withnotes{1}
\def\withindex{0}
\usepackage[T1]{fontenc}
\usepackage[utf8]{inputenc}

%% Eye-candy
\usepackage{lmodern}
\usepackage{xspace}                                     % Smart spacing with \xspace
\usepackage[protrusion=true,expansion=true]{microtype}  % Improve font rendering

% Striking out text
\usepackage[normalem]{ulem}

%% Math
\usepackage{amsfonts,amsmath,amssymb, amsthm, mathtools}
\usepackage{thm-restate}
\usepackage{dsfont} % For the indicator symbol

% Algorithm environment
\usepackage{algorithmicx,algpseudocode,algorithm}

% Colors (with names)
\usepackage[usenames,dvipsnames,table]{xcolor}

% Quotes: \blockquote command
\usepackage{csquotes}

% Relative sizes for text
\usepackage{relsize}

% Bibliography
%\usepackage[numbers]{natbib}

% Required for the table of results
\usepackage{multirow}
\usepackage{chngpage} % allows for temporary adjustment of side margins

% For the commands such as \capitalisewords
\usepackage{mfirstuc}

% Graphics
\usepackage{tikz}
\usetikzlibrary{arrows}
\usetikzlibrary{calc,decorations.pathmorphing,patterns}

% For indexing
\ifnum\withindex=1
  \usepackage{makeidx}
  \usepackage{ifthen}
  \newcommand\indexed[2][]{\ifthenelse{\equal{#1}{}}{#2\index{#2}}{#2\index{#1}}}
  \makeindex %%%% Enable indexing
\fi
%%\usepackage{showidx} % To debug; does not play well with hyperref

% References and links
\usepackage[backref,colorlinks,citecolor=blue,bookmarks=true,linktocpage]{hyperref}
\usepackage{aliascnt}
\usepackage[numbered]{bookmark}

% Full pages
\usepackage{fullpage}

% Titling
\usepackage{titling}

% Compressed lists
\usepackage[shortlabels]{enumitem}
  \setitemize{noitemsep,topsep=3pt,parsep=2pt,partopsep=2pt} % Uncomment for compact item lists
  \setenumerate{itemsep=1pt,topsep=2pt,parsep=2pt,partopsep=2pt}
  \setdescription{itemsep=1pt}
  
% Package for todo notes.
\ifnum\withnotes=1
  \usepackage[colorinlistoftodos,textsize=scriptsize]{todonotes}
\fi

% Verbatim inputs and code
\usepackage{verbatim}

% Resizable parentheses that work (without the space between \left(#1\right)
\usepackage{mleftright} % \mleft( #1 \mright)

\makeatletter
\@ifundefined{theorem}{%
  % Theorems (each with its own style, all same counter). Cf. http://ftp.math.purdue.edu/mirrors/ctan.org/macros/latex/contrib/hyperref/doc/manual.pdf, p.17
  \theoremstyle{plain} %% Style
  	\newtheorem{theorem}{Theorem}%[section]
  	\newaliascnt{coro}{theorem}
  	  \newtheorem{corollary}[coro]{Corollary}
  	\aliascntresetthe{coro}
  	\newaliascnt{lem}{theorem}
  		\newtheorem{lemma}[lem]{Lemma}
  	\aliascntresetthe{lem}
  	\newaliascnt{clm}{theorem}
  		\newtheorem{claim}[clm]{Claim}
	\aliascntresetthe{clm}
	\newaliascnt{fact}{theorem}
 	 	\newtheorem{fact}[theorem]{Fact}
	\aliascntresetthe{fact}
  	\newtheorem*{unnumberedfact}{Fact}
  \newaliascnt{prop}{theorem}
  		\newtheorem{proposition}[prop]{Proposition}
	\aliascntresetthe{prop}
	\newaliascnt{conj}{theorem}
  		\newtheorem{conjecture}[conj]{Conjecture}
	\aliascntresetthe{conj}
  \theoremstyle{remark} %% Style
  	\newtheorem{remark}[theorem]{Remark}
  	\newtheorem{question}[theorem]{Question}
  	\newtheorem*{notation}{Notation}
 	 \newtheorem{example}[theorem]{Example}
  \theoremstyle{definition} %% Style
  	\newaliascnt{defn}{theorem}
 		 \newtheorem{definition}[defn]{Definition}
 	 \aliascntresetthe{defn}
}{}
\makeatother
\providecommand*{\lemautorefname}{Lemma} % For \autoref{} to know the name of lemmas
\providecommand*{\clmautorefname}{Claim}
\providecommand*{\propautorefname}{Proposition}
\providecommand*{\coroautorefname}{Corollary}
\providecommand*{\defnautorefname}{Definition}
\newenvironment{proofof}[1]{\begin{proof}[Proof of {#1}]}{\end{proof}}

%% \email{} command
\providecommand{\email}[1]{\href{mailto:#1}{\nolinkurl{#1}\xspace}}

%% Remarks and notes
\ifnum\withcolors=1
  \newcommand{\new}[1]{{\color{red} {#1}}} % new
  \newcommand{\newer}[1]{{\color{blue} {#1}}} % even newer
  \newcommand{\newest}[1]{{\color{orange} {#1}}} % even even newer
  \newcommand{\newerest}[1]{{\color{blue!10!black!40!green} {#1}}} % you get the idea.
  \newcommand{\ccolor}[1]{{\color{RubineRed}#1}} % Clement
\else
  \newcommand{\new}[1]{{{#1}}}
  \newcommand{\newer}[1]{{{#1}}}
  \newcommand{\newest}[1]{{{#1}}}
  \newcommand{\newerest}[1]{{{#1}}}
  \newcommand{\ccolor}[1]{{#1}}
\fi

\ifnum\withnotes=1
  \newcommand{\cnote}[1]{\par\ccolor{\textbf{C: }\sf #1}} % Clement
  \newcommand{\todonote}[2][]{\todo[size=\scriptsize,color=red!40,#1]{#2}}  
	\newcommand{\questionnote}[2][]{\todo[size=\scriptsize,color=blue!30]{#2}}
	\newcommand{\todonotedone}[2][]{\todo[size=\scriptsize,color=green!40]{$\checkmark$ #2}}
	\newcommand{\todonoteinline}[2][]{\todo[inline,size=\scriptsize,color=orange!40,#1]{#2}}  
  \newcommand{\marginnote}[1]{\todo[color=white,linecolor=black]{{#1}}}
\else
  \newcommand{\cnote}[1]{}
  \newcommand{\todonote}[2][]{\ignore{#2}}
	\newcommand{\questionnote}[2][]{\ignore{#2}}
	\newcommand{\todonotedone}[2][]{\ignore{#2}}
	\newcommand{\todonoteinline}[2][]{\ignore{#2}}
  \newcommand{\marginnote}[1]{\ignore{#1}}
\fi
\newcommand{\ignore}[1]{\leavevmode\unskip} % eat unnecessary spaces before
\newcommand{\cmargin}[1]{\questionnote{\ccolor{#1}}} % Clement

% Shortcuts
\newcommand{\eps}{\ensuremath{\varepsilon}\xspace}
\newcommand{\Algo}{\ensuremath{\mathcal{A}}\xspace} % Algorithm A
\newcommand{\Tester}{\ensuremath{\mathcal{T}}\xspace} % Testing algorithm T
\newcommand{\Learner}{\ensuremath{\mathcal{L}}\xspace} % Learning algorithm L
\newcommand{\property}{\ensuremath{\mathcal{P}}\xspace} % Property P
\newcommand{\class}{\ensuremath{\mathcal{C}}\xspace} % Class C
\newcommand{\eqdef}{\stackrel{\rm def}{=}}
\newcommand{\eqlaw}{\stackrel{\mathcal{L}}{=}}
\newcommand{\accept}{\textsf{ACCEPT}\xspace}
\newcommand{\fail}{\textsf{FAIL}\xspace}
\newcommand{\reject}{\textsf{REJECT}\xspace}
\newcommand{\opt}{{\textsc{opt}}\xspace}
\newcommand{\half}{\frac{1}{2}}
\newcommand{\domain}{\ensuremath{\Omega}\xspace} % Domain of a distribution (default notation)
\newcommand{\distribs}[1]{\Delta\!\left(#1\right)} % Domain of a distribution (default notation)
\newcommand{\yes}{{\sf{}yes}\xspace}
\newcommand{\no}{{\sf{}no}\xspace}
\newcommand{\dyes}{{\cal Y}}
\newcommand{\dno}{{\cal N}}

% Complexity
\newcommand{\littleO}[1]{{o\mleft( #1 \mright)}}
\newcommand{\bigO}[1]{{O\mleft( #1 \mright)}}
\newcommand{\bigOSmall}[1]{{O\big( #1 \big)}}
\newcommand{\bigTheta}[1]{{\Theta\mleft( #1 \mright)}}
\newcommand{\bigOmega}[1]{{\Omega\mleft( #1 \mright)}}
\newcommand{\bigOmegaSmall}[1]{{\Omega\big( #1 \big)}}
\newcommand{\tildeO}[1]{\tilde{O}\mleft( #1 \mright)}
\newcommand{\tildeTheta}[1]{\operatorname{\tilde{\Theta}}\mleft( #1 \mright)}
\newcommand{\tildeOmega}[1]{\operatorname{\tilde{\Omega}}\mleft( #1 \mright)}
\providecommand{\poly}{\operatorname*{poly}}

% Influence
\newcommand{\totinf}[1][f]{{\mathbf{Inf}[#1]}}
\newcommand{\infl}[2][f]{{\mathbf{Inf}_{#1}(#2)}}
\newcommand{\infldeg}[3][f]{{\mathbf{Inf}_{#1}^{#2}(#3)}}

% Sets and indicators
\newcommand{\setOfSuchThat}[2]{ \left\{\; #1 \;\colon\; #2\; \right\} } 			% sets such as "{ elems | condition }"
\newcommand{\indicSet}[1]{\mathds{1}_{#1}}                                              % indicator function
\newcommand{\indic}[1]{\indicSet{\left\{#1\right\}}}                                             % indicator function
\newcommand{\disjunion}{\amalg}%\coprod, \dotcup...

% Distance
\newcommand{\dtv}{\operatorname{d}_{\rm TV}}
\newcommand{\kl}{\operatorname{KL}}
\newcommand{\dhell}{\operatorname{d_{\rm{}H}}}
\newcommand{\hellinger}[2]{{\dhell\mleft({#1, #2}\mright)}}
\newcommand{\kldiv}[2]{{\kl\mleft({#1 \,\|\, #2}\mright)}}
\newcommand{\kolmogorov}[2]{{\operatorname{d_{\rm{}K}}\mleft({#1, #2}\mright)}}
\newcommand{\totalvardistrestr}[3][]{{\dtv^{#1}\mleft({#2, #3}\mright)}}
\newcommand{\totalvardist}[2]{\totalvardistrestr[]{#1}{#2}}
%\newcommand{\chisquarerestr}[3][]{{\operatorname{d}^{#1}_{\chi^2}\mleft({#2 \mid\mid #3}\mright)}}
\newcommand{\chisquare}[2]{{\chi^2\mleft({#1 \mid\mid #2}\mright)}}
\newcommand{\dist}[2]{\operatorname{dist}\mleft({#1, #2}\mright)}

% Restriction (functions, sequences, etc.)
\newcommand\restr[2]{{% we make the whole thing an ordinary symbol
  \left.\kern-\nulldelimiterspace % automatically resize the bar with \right
  #1 % the function
  \vphantom{\big|} % pretend it's a little taller at normal size
  \right|_{#2} % this is the delimiter
  }}

% Probability
\newcommand{\proba}{\Pr}
\newcommand{\probaOf}[1]{\proba\!\left[\, #1\, \right]}
\newcommand{\probaCond}[2]{\proba\!\left[\, #1 \;\middle\vert\; #2\, \right]}
\newcommand{\probaDistrOf}[2]{\proba_{#1}\left[\, #2\, \right]}

% Support of a distribution/function
\newcommand{\supp}[1]{\operatorname{supp}\!\left(#1\right)}

% Expectation & variance
\newcommand{\expect}[1]{\mathbb{E}\!\left[#1\right]}
\newcommand{\expectCond}[2]{\mathbb{E}\!\left[\, #1 \;\middle\vert\; #2\, \right]}
\newcommand{\shortexpect}{\mathbb{E}}
\newcommand{\var}{\operatorname{Var}}

% Distributions
\newcommand{\uniform}{\ensuremath{\mathcal{U}}}
\newcommand{\uniformOn}[1]{\ensuremath{\uniform\!\left( #1 \right) }}
\newcommand{\geom}[1]{\ensuremath{\operatorname{Geom}\!\left( #1 \right)}}
\newcommand{\bernoulli}[1]{\ensuremath{\operatorname{Bern}\!\left( #1 \right)}}
\newcommand{\bern}[2]{\ensuremath{\operatorname{Bern}^{#1}\!\left( #2 \right)}}
\newcommand{\binomial}[2]{\ensuremath{\operatorname{Bin}\!\left( #1, #2 \right)}}
\newcommand{\poisson}[1]{\ensuremath{\operatorname{Poisson}\!\left( #1 \right) }}
\newcommand{\gaussian}[2]{\ensuremath{ \mathcal{N}\!\left(#1,#2\right) }}
\newcommand{\gaussianpdf}[2]{\ensuremath{ g_{#1,#2}}}
\newcommand{\betadistr}[2]{\ensuremath{ \operatorname{Beta}\!\left( #1, #2 \right) }}

% Norms
\newcommand{\norm}[1]{\lVert#1{\rVert}}
\newcommand{\normone}[1]{{\norm{#1}}_1}
\newcommand{\normtwo}[1]{{\norm{#1}}_2}
\newcommand{\norminf}[1]{{\norm{#1}}_\infty}
\newcommand{\abs}[1]{\left\lvert #1 \right\rvert}
\newcommand{\dabs}[1]{\lvert #1 \rvert}
\newcommand{\dotprod}[2]{ \left\langle #1,\xspace #2 \right\rangle } 			% <a,b>
\newcommand{\ip}[2]{\dotprod{#1}{#2}} 			% shortcut

\newcommand{\vect}[1]{\mathbf{#1}} 			% shortcut

% Ceiling and floor
\newcommand{\clg}[1]{\left\lceil #1 \right\rceil}
\newcommand{\flr}[1]{\left\lfloor #1 \right\rfloor}

% Common sets
\newcommand{\R}{\ensuremath{\mathbb{R}}\xspace}
\newcommand{\C}{\ensuremath{\mathbb{C}}\xspace}
\newcommand{\Q}{\ensuremath{\mathbb{Q}}\xspace}
\newcommand{\Z}{\ensuremath{\mathbb{Z}}\xspace}
\newcommand{\N}{\ensuremath{\mathbb{N}}\xspace}
\newcommand{\cont}[1]{\ensuremath{\mathcal{C}^{#1}}}

% Oracles and variants
\newcommand{\ICOND}{{\sf INTCOND}\xspace}
\newcommand{\EVAL}{{\sf EVAL}\xspace}
\newcommand{\CDFEVAL}{{\sf CEVAL}\xspace}
\newcommand{\STAT}{{\sf STAT}\xspace}
\newcommand{\SAMP}{{\sf SAMP}\xspace}
\newcommand{\COND}{{\sf COND}\xspace}
\newcommand{\PCOND}{{\sf PAIRCOND}\xspace}
\newcommand{\ORACLE}{{\sf ORACLE}\xspace}

%% Terminology
\newcommand{\pdfsamp}{dual\xspace}
\newcommand{\cdfsamp}{cumulative dual\xspace}
\newcommand{\Pdfsamp}{\expandafter\capitalisewords\expandafter{\pdfsamp}}
\newcommand{\Cdfsamp}{\expandafter\capitalisewords\expandafter{\cdfsamp}}

% L_p norms
\newcommand{\lp}[1][1]{\ell_{#1}}

% Convolution
\DeclareMathOperator{\convolution}{\ast}

%% Terminology
\newcommand{\D}{\ensuremath{D}}
\newcommand{\distrD}{\ensuremath{\mathcal{D}}}
\newcommand{\birge}[2][\D]{\Phi_{#2}(#1)}
\newcommand{\iid}{i.i.d.\xspace}

% Sign
\DeclareMathOperator{\sign}{sgn}

%% Roman numerals
\makeatletter
\newcommand{\rom}[1]{\romannumeral #1}
\newcommand{\Rom}[1]{\expandafter\@slowromancap\romannumeral #1@}
\newcommand{\century}[2][th]{\Rom{#2}\textsuperscript{#1}}
\makeatother

% Hyperref and \autoref{} -- names
\renewcommand{\sectionautorefname}{Section} % To have "Section 5" instead of "section 5" with \autoref{}
\renewcommand{\chapterautorefname}{Chapter} % To have "Chapter 5" instead of "chapter 5" with \autoref{}
\renewcommand{\subsectionautorefname}{Section} % To have "Section 5" instead of "subsection 5" with \autoref{}
\renewcommand{\subsubsectionautorefname}{Section} % To have "Section 5" instead of "subsubsection 5" with \autoref{}
\def\algorithmautorefname{Algorithm}


%%%%%%%%%%%%%%%%%%%%%%%%%%%%%%%%%%%%%%%%%%%%%%%%%%%%%%%%%%%%%%%%%
% Add author and title info to PDF (and handles multiple authors)
%%%%%%%%%%%%%%%%%%%%%%%%%%%%%%%%%%%%%%%%%%%%%%%%%%%%%%%%%%%%%%%%%
\makeatletter
  \AtBeginDocument{
  \begingroup
  \toks@={}%
  \toksdef\toks@@=2 %
  \toks@@={}%
  \long\def\@ReturnFiFi#1#2\fi\fi{\fi\fi#1}%
  \def\scan@author#1#2 \and#3\@nil{%
  \ifx\\#3\\%
    \ifcase#1 %
      \toks@={#2}%
    \else
      \ifnum#1>1 %
        \toks@=\expandafter{%
          \the\expandafter\toks@\expandafter,\expandafter\space
          \the\toks@@
        }%
      \fi
      \toks@=\expandafter{\the\toks@\space and #2}%
    \fi
    \else
      \ifcase#1 %
        \toks@={#2}%
        \@ReturnFiFi{%
          \scan@author1#3\@nil
        }%
      \else
        \ifnum#1>1 %
          \toks@=\expandafter{%
            \the\expandafter\toks@\expandafter,\expandafter\space
            \the\toks@@
          }%
      \fi
      \toks@@={#2}%
      \@ReturnFiFi{%
        \scan@author2#3\@nil
      }%
    \fi
  \fi
  }%
  \expandafter\expandafter\expandafter\scan@author
  \expandafter\expandafter\expandafter0%
  \expandafter\@author\space\and\@nil
  \edef\x{\endgroup
  \noexpand\hypersetup{pdfauthor={\the\toks@}}%
  }%
  \x
  }
\makeatother


\newcommand{\bennettfunc}{h}
\newcommand{\bennett}[1]{\bennettfunc\mleft(#1\mright)}

\title{A short note on Poisson tail bounds}
\date{February, 2016}

\begin{document}
\begin{flushleft}\sf\footnotesize
\makeatletter
\@date~- \today \hfill \@title
\makeatother
\end{flushleft}
\vspace{5mm}

The goal of this short note is to provide a proof and references for the ``folklore fact'' that Poisson random variables enjoy good concentration bounds -- namely, subexponential. Thanks to \href{http://www.gautamkamath.com/}{Gautam Kamath} for bringing the topic to my attention, and making me realize I originally had neither of the two.\bigskip

\noindent Let $\bennettfunc\colon[-1,\infty) \to \R$ be the function defined by $h(u)\eqdef 2\frac{(1+u)\ln(1+u)-u}{u^2}$.

\begin{theorem}\label{theo:main:poisson:bounds}
Let $X\sim\poisson{\lambda}$, for some parameter $\lambda > 0$. Then, for any $x>0$, we have
\begin{equation}\label{eq:poisson:upper:tail}
    \probaOf{ X \geq \lambda + x} \leq e^{-\frac{x^2}{2\lambda}\bennett{\frac{x}{\lambda}}}
\end{equation}
and, for any $0<x< \lambda$,
\begin{equation}\label{eq:poisson:lower:tail}
  \probaOf{ X \leq \lambda - x} \leq e^{-\frac{x^2}{2\lambda}\bennett{-\frac{x}{\lambda}}}.
\end{equation}
In particular, this implies that $\probaOf{ X \geq \lambda + x},\probaOf{ X \leq \lambda - x} \leq e^{-\frac{x^2}{\lambda+x}}$, for $x>0$; from which
\begin{equation}\label{eq:poisson:both:tail}
  \probaOf{ \abs{X -\lambda} \geq x} \leq 2e^{-\frac{x^2}{2(\lambda+x)}}, \qquad x>0.
\end{equation}
\end{theorem}
\begin{proof}
Equations~\eqref{eq:poisson:upper:tail} and~\eqref{eq:poisson:lower:tail} are proven in~\autoref{fact:poisson:upper:tail} and~\autoref{fact:poisson:lower:tail}, respectively. We show how they imply~\eqref{eq:poisson:both:tail}.

\noindent By~\autoref{fact:bennett:function:ii}, it is the case that,  for every $x>0$, $\bennett{\frac{x}{\lambda}} \geq \frac{1}{1+\frac{x}{\lambda}}$, or equivalently
$
\frac{x^2}{2\lambda} \bennett{\frac{x}{\lambda}} \geq \frac{x^2}{2(\lambda+x)}
$. Thus, from~\eqref{eq:poisson:upper:tail} we get $\probaOf{ X \geq \lambda + x} \leq \exp(-\frac{x^2}{2\lambda}\bennett{\frac{x}{\lambda}}) \leq \exp(-\frac{x^2}{2(\lambda+x)})$.

\noindent Similarly, for any $0<x<\lambda$ we have $\frac{x^2}{2\lambda} > \frac{x^2}{2(\lambda+x)}$, which with~\eqref{eq:poisson:lower:tail} and~\autoref{fact:bennett:function} implies
$
\probaOf{ X \leq \lambda - x} \leq \exp(-\frac{x^2}{2\lambda}\bennett{-\frac{x}{\lambda}}) \leq \exp(-\frac{x^2}{2\lambda}\bennett{0}) = \exp(-\frac{x^2}{2\lambda}) \leq \exp(-\frac{x^2}{2(\lambda+x)})
$.
\end{proof}

\noindent Thus, we are left with proving \autoref{fact:poisson:upper:tail} and~\autoref{fact:poisson:lower:tail}, which we do next.

\section{Establishing~\eqref{eq:poisson:upper:tail} and~\eqref{eq:poisson:lower:tail}}

\begin{fact}\label{fact:bennett:function}
  We have $\bennett{-1}=2$, $\bennett{0}=1$, and $\bennettfunc$ decreasing on $[-1,\infty)$ with $\lim_{u\to\infty} \bennett{u}=0$. In particular, $\bennettfunc\geq 0$.
\end{fact}
\begin{proof}
  The first two properties are immediate by continuity, as, for $u\notin\{-1,0\}$,
  \begin{align*}
      \bennett{u} &= 2\frac{(1+u)\ln(1+u)-u}{u^2} \xrightarrow[u\to-1]{} 2\frac{0-(-1)}{(-1)^2} = 2 \\
      \bennett{u} &= 2\frac{(1+u)\ln(1+u)-u}{u^2} = 2\frac{(1+u)(u-\frac{u^2}{2}+o(u^2))-u}{u^2}
                 = 2\frac{\frac{u^2}{2}+o(u^2)}{u^2} \xrightarrow[u\to0]{} 1
  \end{align*}
  The third property follows from differentiating the function on $(-1,0)\cup(0,\infty)$ and showing its derivative is negative; or, more cleverly, following~\cite[Exercise 14, (ii)]{Pollard:15}. The fourth (which together with the third implies the last) directly comes from observing that $\bennett{u} \operatorname*{\sim}_{u\to\infty} \frac{2\ln u}{u}$.
\end{proof}

\begin{fact}\label{fact:bennett:function:ii}
  For any $u\geq 0$, we have $\bennett{u} \geq \frac{1}{1+u}$.
\end{fact}
\begin{proof}
  Consider the function $g\colon [0,\infty) \to \R$ defined by $g(u) = (1+u)\bennett{u}$. We then have $g(0)=1$, and $g(u) \operatorname*{\sim}_{u\to\infty} 2\ln u \xrightarrow[u\to\infty]{} \infty$. Moreover, by differentiation(s) (and tedious computations), one can show that $g$ is increasing on $[0,\infty)$, which implies the claim.
\end{proof}

We follow the outline of~\cite[Exercise 15]{Pollard:15}. For a random variable $X$, we denote by $M$ its moment-generating function, i.e. $M_X\colon \theta\in\R \mapsto \expect{e^{\theta X}}$ (provided it is well-defined). In what follows, $X$ is a random variable following a $\poisson{\lambda}$ distribution.

\begin{fact}\label{fact:mgf:poisson}
  We have $M_X(\theta) = e^{\lambda(e^\theta-1)}$ for every $\theta \in \R$.
\end{fact}
\begin{proof}
This is a standard fact, we give the derivation for completeness. For any $\theta\in\R$,
\[
    M_X(\theta) = \expect{e^{\theta X}} = e^{-\lambda}\sum_{n=0}^\infty e^{\theta n}\frac{\lambda^n}{n!} = e^{-\lambda}\sum_{n=0}^\infty \frac{(e^\theta \lambda)^n}{n!} = e^{-\lambda} e^{e^\theta \lambda}
    = e^{\lambda(e^\theta-1)}.
\]
\end{proof}

\begin{fact}\label{fact:poisson:upper:tail}
  For any $x>0$, $\probaOf{ X \geq \lambda + x} \leq e^{-\frac{x^2}{2\lambda}\bennett{\frac{x}{\lambda}}}$.
\end{fact}
\begin{proof}
Fix $x> 0$. For any $\theta\in\R$,
\begin{align*}
    \probaOf{ X \geq \lambda + x}
    &= \probaOf{ e^{\theta X} \geq e^{\theta(\lambda + x)} }
    = \probaOf{ e^{\theta(X - \lambda -x)} \geq 1}
    \leq \expect{ e^{\theta(X - \lambda -x)} } 
\end{align*}
recalling that if $Y$ is a discrete random variable taking values in $\N$, $\probaOf{Y > 0}=\probaOf{Y \geq 1} = \sum_{n=1}^\infty \probaOf{Y = n} \leq \sum_{n=1}^\infty n\probaOf{Y = n} = \expect{Y}$. Rearranging the terms and taking the infimum over all $\theta > 0$, we have
\begin{align*}
    \probaOf{ X \geq \lambda + x}
    &\leq \inf_{\theta > 0}\expect{ e^{\theta X} }e^{-\theta(\lambda+x)} = \inf_{\theta > 0}e^{\lambda(e^\theta-1)}e^{-\theta(\lambda+x)} \tag{\autoref{fact:mgf:poisson}} \\
    &= \inf_{\theta > 0}e^{\lambda(e^\theta-1) -\theta(\lambda+x)}
    = e^{\inf_{\theta > 0}( \lambda(e^\theta-1) -\theta(\lambda+x))}.
\end{align*}
It is a simple matter of calculus to find that $\inf_{\theta > 0}( \lambda(e^\theta-1) -\theta(\lambda+x))$ is attained for $\theta^\ast \eqdef \ln(1+\frac{x}{\lambda}) > 0$, from which
\begin{align*}
    \probaOf{ X \geq \lambda + x}
    &\leq e^{\lambda(e^{\theta^\ast}-1) -\theta^\ast(\lambda+x)}
    = e^{-\lambda((1+\frac{x}{\lambda})\ln(1+\frac{x}{\lambda})-\frac{x}{\lambda})} 
    = e^{-\frac{x^2}{2\lambda}\bennett{\frac{x}{\lambda}}} 
\end{align*}
as claimed.
\end{proof}


\begin{fact}\label{fact:poisson:lower:tail}
  For any $0 < x< \lambda$, $\probaOf{ X \leq \lambda - x} \leq e^{-\frac{x^2}{2\lambda}\bennett{-\frac{x}{\lambda}}}\leq e^{-\frac{x^2}{2\lambda}}$.
\end{fact}
\begin{proof}
Fix $0 < x< \lambda$. As before, for any $\theta\in\R$,
\begin{align*}
    \probaOf{ X \leq \lambda - x}
    &= \probaOf{ e^{\theta X} \leq e^{\theta(\lambda - x)} }
    = \probaOf{ e^{\theta(\lambda -x - X)} \geq 1}
    \leq \expect{ e^{-\theta X} } e^{\theta(\lambda -x) }. 
\end{align*}
Rearranging the terms and taking the infimum over all $\theta > 0$, we have
\begin{align*}
    \probaOf{ X \leq \lambda - x}
    &\leq \inf_{\theta > 0}\expect{ e^{-\theta X} }e^{\theta(\lambda-x)} = \inf_{\theta > 0}e^{\lambda(e^{-\theta}-1)}e^{\theta(\lambda-x)} \tag{\autoref{fact:mgf:poisson}} \\
    &= e^{\inf_{\theta > 0}( \lambda(e^{-\theta}-1) + \theta(\lambda-x))}.
\end{align*}
It is again straightforward to check, e.g. by differentiation, that $\inf_{\theta > 0}( \lambda(e^{-\theta}-1) + \theta(\lambda-x))$ is attained for $\theta^\ast \eqdef -\ln(1-\frac{x}{\lambda}) > 0$, from which
\begin{align*}
    \probaOf{ X \leq \lambda - x}
    &\leq e^{\lambda(e^{-\theta^\ast}-1) +\theta^\ast(\lambda-x)}
    = e^{-x -(\lambda-x)\ln(1-\frac{x}{\lambda})} 
    = e^{-\lambda( (1-\frac{x}{\lambda})\ln(1-\frac{x}{\lambda}) + \frac{x}{\lambda})} 
    = e^{-\frac{x^2}{2\lambda}\bennett{-\frac{x}{\lambda}}} 
\end{align*}
as claimed. The last step is to observe that, by~\autoref{fact:bennett:function}, $e^{-\frac{x^2}{2\lambda}\bennett{-\frac{x}{\lambda}}} \leq e^{-\frac{x^2}{2\lambda}\bennett{0}} = e^{-\frac{x^2}{2\lambda}}$.
\end{proof}

%%%%%%%%%%%%%%%%%%%%%%%%%%%%%%%%%%%%%%%%%%%%%%%%%%%%%%%%%%%%%%%%%%%%%%%%%%%%%%%%%%%%%%%%%%
\section{An alternative proof of~\eqref{eq:poisson:upper:tail}}
\footnotetext{This approach is inspired by~\cite[Exercise 16]{Pollard:15}).}
Recall that if $(Y^{(n)})_{n\geq 1}$ is a sequence of independent random variables such that $Y^{(n)}$ follows a $\binomial{n}{\frac{\lambda}{n}}$ distribution, then $(Y^{(n)})_{n\geq 1}$ converges in law to $X$, a random variable with $\poisson{\lambda}$ distribution.\footnotemark{} In particular, since convergence in law corresponds to pointwise convergence of distribution functions, this implies that, for any $t\in \R$,
\begin{equation}\label{eq:convergence:law:poisson}
  \probaOf{Y^{(n)} \geq t } \xrightarrow[n\to\infty]{}  \probaOf{X \geq t }.
\end{equation}

For any fixed $n\geq 1$, we can by definition write $Y^{(n)}$ as $Y^{(n)}=\sum_{k=1}^n Y_k^{(n)}$, where $Y^{(n)}_1,\dots,Y^{(n)}_n$ are i.i.d. random variables with $\bernoulli{\frac{\lambda}{n}}$ distribution. Note that $\expect{Y^{(n)}}=\lambda$ and $\var[Y^{(n)}] = \lambda(1-\frac{\lambda}{n}) \leq \lambda$. As
$\expect{Y_k^{(n)}} = \frac{\lambda}{n}$ and $\dabs{Y_k^{(n)}}\leq 1$ for all $1\leq k\leq n$, we can apply Bennett's inequality (\cite[Chapter 2]{Boucheron:13},\cite[Chapter 2.5]{Pollard:15}), to obtain, for any $t\geq 0$,
\[
    \probaOf{ Y^{(n)} \geq \lambda + x }
    = \probaOf{ Y^{(n)} \geq \expect{Y^{(n)}} + x }
    \leq e^{ -\frac{x^2}{2\lambda} \bennett{\frac{x}{\lambda}} }
\]
Taking the limit as $n$ goes to $\infty$, we obtain by~\eqref{eq:convergence:law:poisson} that $\probaOf{X \geq \lambda + x } \leq e^{ -\frac{x^2}{2\lambda} \bennett{\frac{x}{\lambda}} }$, re-establishing~\eqref{eq:poisson:upper:tail}.

\begin{remark}
We note that a qualitatively similar statement (yet quantitatively weaker) can be obtained by observing that Poisson distributions are in particular (discrete) log-concave, and that any log-concave (discrete or continuous) has subexponential tail~\cite{An:95}.
\end{remark}

\begin{remark}
As another way to establish the result, we refer the reader to~\cite[Proposition 11.15]{Gol:17}, where bounds on individual summands of the Poisson tails are obtained. From there, one can attempt to derive~\autoref{theo:main:poisson:bounds}, specifically~\eqref{eq:poisson:both:tail}.
\end{remark}
%\nocite{*}
  \bibliographystyle{alpha}
  \bibliography{poissonconcentration} 
\end{document}
