\documentclass[10pt]{article}
\def\withcolors{1}
\def\withnotes{1}
\def\withindex{0}
\usepackage[T1]{fontenc}
\usepackage[utf8]{inputenc}

%% Eye-candy
\usepackage{lmodern}
\usepackage{xspace}                                     % Smart spacing with \xspace
\usepackage[protrusion=true,expansion=true]{microtype}  % Improve font rendering

% Striking out text
\usepackage[normalem]{ulem}

%% Math
\usepackage{amsfonts,amsmath,amssymb, amsthm, mathtools}
\usepackage{thm-restate}
\usepackage{dsfont} % For the indicator symbol

% Algorithm environment
\usepackage{algorithmicx,algpseudocode,algorithm}

% Colors (with names)
\usepackage[usenames,dvipsnames,table]{xcolor}

% Quotes: \blockquote command
\usepackage{csquotes}

% Relative sizes for text
\usepackage{relsize}

% Bibliography
%\usepackage[numbers]{natbib}

% Required for the table of results
\usepackage{multirow}
\usepackage{chngpage} % allows for temporary adjustment of side margins

% For the commands such as \capitalisewords
\usepackage{mfirstuc}

% Graphics
\usepackage{tikz}
\usetikzlibrary{arrows}
\usetikzlibrary{calc,decorations.pathmorphing,patterns}

% For indexing
\ifnum\withindex=1
  \usepackage{makeidx}
  \usepackage{ifthen}
  \newcommand\indexed[2][]{\ifthenelse{\equal{#1}{}}{#2\index{#2}}{#2\index{#1}}}
  \makeindex %%%% Enable indexing
\fi
%%\usepackage{showidx} % To debug; does not play well with hyperref

% References and links
\usepackage[backref,colorlinks,citecolor=blue,bookmarks=true,linktocpage]{hyperref}
\usepackage{aliascnt}
\usepackage[numbered]{bookmark}

% Full pages
\usepackage{fullpage}

% Titling
\usepackage{titling}

% Compressed lists
\usepackage[shortlabels]{enumitem}
  \setitemize{noitemsep,topsep=3pt,parsep=2pt,partopsep=2pt} % Uncomment for compact item lists
  \setenumerate{itemsep=1pt,topsep=2pt,parsep=2pt,partopsep=2pt}
  \setdescription{itemsep=1pt}
  
% Package for todo notes.
\ifnum\withnotes=1
  \usepackage[colorinlistoftodos,textsize=scriptsize]{todonotes}
\fi

% Verbatim inputs and code
\usepackage{verbatim}

% Resizable parentheses that work (without the space between \left(#1\right)
\usepackage{mleftright} % \mleft( #1 \mright)

\makeatletter
\@ifundefined{theorem}{%
  % Theorems (each with its own style, all same counter). Cf. http://ftp.math.purdue.edu/mirrors/ctan.org/macros/latex/contrib/hyperref/doc/manual.pdf, p.17
  \theoremstyle{plain} %% Style
  	\newtheorem{theorem}{Theorem}%[section]
  	\newaliascnt{coro}{theorem}
  	  \newtheorem{corollary}[coro]{Corollary}
  	\aliascntresetthe{coro}
  	\newaliascnt{lem}{theorem}
  		\newtheorem{lemma}[lem]{Lemma}
  	\aliascntresetthe{lem}
  	\newaliascnt{clm}{theorem}
  		\newtheorem{claim}[clm]{Claim}
	\aliascntresetthe{clm}
	\newaliascnt{fact}{theorem}
 	 	\newtheorem{fact}[theorem]{Fact}
	\aliascntresetthe{fact}
  	\newtheorem*{unnumberedfact}{Fact}
  \newaliascnt{prop}{theorem}
  		\newtheorem{proposition}[prop]{Proposition}
	\aliascntresetthe{prop}
	\newaliascnt{conj}{theorem}
  		\newtheorem{conjecture}[conj]{Conjecture}
	\aliascntresetthe{conj}
  \theoremstyle{remark} %% Style
  	\newtheorem{remark}[theorem]{Remark}
  	\newtheorem{question}[theorem]{Question}
  	\newtheorem*{notation}{Notation}
 	 \newtheorem{example}[theorem]{Example}
  \theoremstyle{definition} %% Style
  	\newaliascnt{defn}{theorem}
 		 \newtheorem{definition}[defn]{Definition}
 	 \aliascntresetthe{defn}
}{}
\makeatother
\providecommand*{\lemautorefname}{Lemma} % For \autoref{} to know the name of lemmas
\providecommand*{\clmautorefname}{Claim}
\providecommand*{\propautorefname}{Proposition}
\providecommand*{\coroautorefname}{Corollary}
\providecommand*{\defnautorefname}{Definition}
\newenvironment{proofof}[1]{\begin{proof}[Proof of {#1}]}{\end{proof}}

%% \email{} command
\providecommand{\email}[1]{\href{mailto:#1}{\nolinkurl{#1}\xspace}}

%% Remarks and notes
\ifnum\withcolors=1
  \newcommand{\new}[1]{{\color{red} {#1}}} % new
  \newcommand{\newer}[1]{{\color{blue} {#1}}} % even newer
  \newcommand{\newest}[1]{{\color{orange} {#1}}} % even even newer
  \newcommand{\newerest}[1]{{\color{blue!10!black!40!green} {#1}}} % you get the idea.
  \newcommand{\ccolor}[1]{{\color{RubineRed}#1}} % Clement
\else
  \newcommand{\new}[1]{{{#1}}}
  \newcommand{\newer}[1]{{{#1}}}
  \newcommand{\newest}[1]{{{#1}}}
  \newcommand{\newerest}[1]{{{#1}}}
  \newcommand{\ccolor}[1]{{#1}}
\fi

\ifnum\withnotes=1
  \newcommand{\cnote}[1]{\par\ccolor{\textbf{C: }\sf #1}} % Clement
  \newcommand{\todonote}[2][]{\todo[size=\scriptsize,color=red!40,#1]{#2}}  
	\newcommand{\questionnote}[2][]{\todo[size=\scriptsize,color=blue!30]{#2}}
	\newcommand{\todonotedone}[2][]{\todo[size=\scriptsize,color=green!40]{$\checkmark$ #2}}
	\newcommand{\todonoteinline}[2][]{\todo[inline,size=\scriptsize,color=orange!40,#1]{#2}}  
  \newcommand{\marginnote}[1]{\todo[color=white,linecolor=black]{{#1}}}
\else
  \newcommand{\cnote}[1]{}
  \newcommand{\todonote}[2][]{\ignore{#2}}
	\newcommand{\questionnote}[2][]{\ignore{#2}}
	\newcommand{\todonotedone}[2][]{\ignore{#2}}
	\newcommand{\todonoteinline}[2][]{\ignore{#2}}
  \newcommand{\marginnote}[1]{\ignore{#1}}
\fi
\newcommand{\ignore}[1]{\leavevmode\unskip} % eat unnecessary spaces before
\newcommand{\cmargin}[1]{\questionnote{\ccolor{#1}}} % Clement

% Shortcuts
\newcommand{\eps}{\ensuremath{\varepsilon}\xspace}
\newcommand{\Algo}{\ensuremath{\mathcal{A}}\xspace} % Algorithm A
\newcommand{\Tester}{\ensuremath{\mathcal{T}}\xspace} % Testing algorithm T
\newcommand{\Learner}{\ensuremath{\mathcal{L}}\xspace} % Learning algorithm L
\newcommand{\property}{\ensuremath{\mathcal{P}}\xspace} % Property P
\newcommand{\class}{\ensuremath{\mathcal{C}}\xspace} % Class C
\newcommand{\eqdef}{\stackrel{\rm def}{=}}
\newcommand{\eqlaw}{\stackrel{\mathcal{L}}{=}}
\newcommand{\accept}{\textsf{ACCEPT}\xspace}
\newcommand{\fail}{\textsf{FAIL}\xspace}
\newcommand{\reject}{\textsf{REJECT}\xspace}
\newcommand{\opt}{{\textsc{opt}}\xspace}
\newcommand{\half}{\frac{1}{2}}
\newcommand{\domain}{\ensuremath{\Omega}\xspace} % Domain of a distribution (default notation)
\newcommand{\distribs}[1]{\Delta\!\left(#1\right)} % Domain of a distribution (default notation)
\newcommand{\yes}{{\sf{}yes}\xspace}
\newcommand{\no}{{\sf{}no}\xspace}
\newcommand{\dyes}{{\cal Y}}
\newcommand{\dno}{{\cal N}}

% Complexity
\newcommand{\littleO}[1]{{o\mleft( #1 \mright)}}
\newcommand{\bigO}[1]{{O\mleft( #1 \mright)}}
\newcommand{\bigOSmall}[1]{{O\big( #1 \big)}}
\newcommand{\bigTheta}[1]{{\Theta\mleft( #1 \mright)}}
\newcommand{\bigOmega}[1]{{\Omega\mleft( #1 \mright)}}
\newcommand{\bigOmegaSmall}[1]{{\Omega\big( #1 \big)}}
\newcommand{\tildeO}[1]{\tilde{O}\mleft( #1 \mright)}
\newcommand{\tildeTheta}[1]{\operatorname{\tilde{\Theta}}\mleft( #1 \mright)}
\newcommand{\tildeOmega}[1]{\operatorname{\tilde{\Omega}}\mleft( #1 \mright)}
\providecommand{\poly}{\operatorname*{poly}}

% Influence
\newcommand{\totinf}[1][f]{{\mathbf{Inf}[#1]}}
\newcommand{\infl}[2][f]{{\mathbf{Inf}_{#1}(#2)}}
\newcommand{\infldeg}[3][f]{{\mathbf{Inf}_{#1}^{#2}(#3)}}

% Sets and indicators
\newcommand{\setOfSuchThat}[2]{ \left\{\; #1 \;\colon\; #2\; \right\} } 			% sets such as "{ elems | condition }"
\newcommand{\indicSet}[1]{\mathds{1}_{#1}}                                              % indicator function
\newcommand{\indic}[1]{\indicSet{\left\{#1\right\}}}                                             % indicator function
\newcommand{\disjunion}{\amalg}%\coprod, \dotcup...

% Distance
\newcommand{\dtv}{\operatorname{d}_{\rm TV}}
\newcommand{\kl}{\operatorname{KL}}
\newcommand{\dhell}{\operatorname{d_{\rm{}H}}}
\newcommand{\hellinger}[2]{{\dhell\mleft({#1, #2}\mright)}}
\newcommand{\kldiv}[2]{{\kl\mleft({#1 \,\|\, #2}\mright)}}
\newcommand{\kolmogorov}[2]{{\operatorname{d_{\rm{}K}}\mleft({#1, #2}\mright)}}
\newcommand{\totalvardistrestr}[3][]{{\dtv^{#1}\mleft({#2, #3}\mright)}}
\newcommand{\totalvardist}[2]{\totalvardistrestr[]{#1}{#2}}
%\newcommand{\chisquarerestr}[3][]{{\operatorname{d}^{#1}_{\chi^2}\mleft({#2 \mid\mid #3}\mright)}}
\newcommand{\chisquare}[2]{{\chi^2\mleft({#1 \mid\mid #2}\mright)}}
\newcommand{\dist}[2]{\operatorname{dist}\mleft({#1, #2}\mright)}

% Restriction (functions, sequences, etc.)
\newcommand\restr[2]{{% we make the whole thing an ordinary symbol
  \left.\kern-\nulldelimiterspace % automatically resize the bar with \right
  #1 % the function
  \vphantom{\big|} % pretend it's a little taller at normal size
  \right|_{#2} % this is the delimiter
  }}

% Probability
\newcommand{\proba}{\Pr}
\newcommand{\probaOf}[1]{\proba\!\left[\, #1\, \right]}
\newcommand{\probaCond}[2]{\proba\!\left[\, #1 \;\middle\vert\; #2\, \right]}
\newcommand{\probaDistrOf}[2]{\proba_{#1}\left[\, #2\, \right]}

% Support of a distribution/function
\newcommand{\supp}[1]{\operatorname{supp}\!\left(#1\right)}

% Expectation & variance
\newcommand{\expect}[1]{\mathbb{E}\!\left[#1\right]}
\newcommand{\expectCond}[2]{\mathbb{E}\!\left[\, #1 \;\middle\vert\; #2\, \right]}
\newcommand{\shortexpect}{\mathbb{E}}
\newcommand{\var}{\operatorname{Var}}

% Distributions
\newcommand{\uniform}{\ensuremath{\mathcal{U}}}
\newcommand{\uniformOn}[1]{\ensuremath{\uniform\!\left( #1 \right) }}
\newcommand{\geom}[1]{\ensuremath{\operatorname{Geom}\!\left( #1 \right)}}
\newcommand{\bernoulli}[1]{\ensuremath{\operatorname{Bern}\!\left( #1 \right)}}
\newcommand{\bern}[2]{\ensuremath{\operatorname{Bern}^{#1}\!\left( #2 \right)}}
\newcommand{\binomial}[2]{\ensuremath{\operatorname{Bin}\!\left( #1, #2 \right)}}
\newcommand{\poisson}[1]{\ensuremath{\operatorname{Poisson}\!\left( #1 \right) }}
\newcommand{\gaussian}[2]{\ensuremath{ \mathcal{N}\!\left(#1,#2\right) }}
\newcommand{\gaussianpdf}[2]{\ensuremath{ g_{#1,#2}}}
\newcommand{\betadistr}[2]{\ensuremath{ \operatorname{Beta}\!\left( #1, #2 \right) }}

% Norms
\newcommand{\norm}[1]{\lVert#1{\rVert}}
\newcommand{\normone}[1]{{\norm{#1}}_1}
\newcommand{\normtwo}[1]{{\norm{#1}}_2}
\newcommand{\norminf}[1]{{\norm{#1}}_\infty}
\newcommand{\abs}[1]{\left\lvert #1 \right\rvert}
\newcommand{\dabs}[1]{\lvert #1 \rvert}
\newcommand{\dotprod}[2]{ \left\langle #1,\xspace #2 \right\rangle } 			% <a,b>
\newcommand{\ip}[2]{\dotprod{#1}{#2}} 			% shortcut

\newcommand{\vect}[1]{\mathbf{#1}} 			% shortcut

% Ceiling and floor
\newcommand{\clg}[1]{\left\lceil #1 \right\rceil}
\newcommand{\flr}[1]{\left\lfloor #1 \right\rfloor}

% Common sets
\newcommand{\R}{\ensuremath{\mathbb{R}}\xspace}
\newcommand{\C}{\ensuremath{\mathbb{C}}\xspace}
\newcommand{\Q}{\ensuremath{\mathbb{Q}}\xspace}
\newcommand{\Z}{\ensuremath{\mathbb{Z}}\xspace}
\newcommand{\N}{\ensuremath{\mathbb{N}}\xspace}
\newcommand{\cont}[1]{\ensuremath{\mathcal{C}^{#1}}}

% Oracles and variants
\newcommand{\ICOND}{{\sf INTCOND}\xspace}
\newcommand{\EVAL}{{\sf EVAL}\xspace}
\newcommand{\CDFEVAL}{{\sf CEVAL}\xspace}
\newcommand{\STAT}{{\sf STAT}\xspace}
\newcommand{\SAMP}{{\sf SAMP}\xspace}
\newcommand{\COND}{{\sf COND}\xspace}
\newcommand{\PCOND}{{\sf PAIRCOND}\xspace}
\newcommand{\ORACLE}{{\sf ORACLE}\xspace}

%% Terminology
\newcommand{\pdfsamp}{dual\xspace}
\newcommand{\cdfsamp}{cumulative dual\xspace}
\newcommand{\Pdfsamp}{\expandafter\capitalisewords\expandafter{\pdfsamp}}
\newcommand{\Cdfsamp}{\expandafter\capitalisewords\expandafter{\cdfsamp}}

% L_p norms
\newcommand{\lp}[1][1]{\ell_{#1}}

% Convolution
\DeclareMathOperator{\convolution}{\ast}

%% Terminology
\newcommand{\D}{\ensuremath{D}}
\newcommand{\distrD}{\ensuremath{\mathcal{D}}}
\newcommand{\birge}[2][\D]{\Phi_{#2}(#1)}
\newcommand{\iid}{i.i.d.\xspace}

% Sign
\DeclareMathOperator{\sign}{sgn}

%% Roman numerals
\makeatletter
\newcommand{\rom}[1]{\romannumeral #1}
\newcommand{\Rom}[1]{\expandafter\@slowromancap\romannumeral #1@}
\newcommand{\century}[2][th]{\Rom{#2}\textsuperscript{#1}}
\makeatother

% Hyperref and \autoref{} -- names
\renewcommand{\sectionautorefname}{Section} % To have "Section 5" instead of "section 5" with \autoref{}
\renewcommand{\chapterautorefname}{Chapter} % To have "Chapter 5" instead of "chapter 5" with \autoref{}
\renewcommand{\subsectionautorefname}{Section} % To have "Section 5" instead of "subsection 5" with \autoref{}
\renewcommand{\subsubsectionautorefname}{Section} % To have "Section 5" instead of "subsubsection 5" with \autoref{}
\def\algorithmautorefname{Algorithm}


%%%%%%%%%%%%%%%%%%%%%%%%%%%%%%%%%%%%%%%%%%%%%%%%%%%%%%%%%%%%%%%%%
% Add author and title info to PDF (and handles multiple authors)
%%%%%%%%%%%%%%%%%%%%%%%%%%%%%%%%%%%%%%%%%%%%%%%%%%%%%%%%%%%%%%%%%
\makeatletter
  \AtBeginDocument{
  \begingroup
  \toks@={}%
  \toksdef\toks@@=2 %
  \toks@@={}%
  \long\def\@ReturnFiFi#1#2\fi\fi{\fi\fi#1}%
  \def\scan@author#1#2 \and#3\@nil{%
  \ifx\\#3\\%
    \ifcase#1 %
      \toks@={#2}%
    \else
      \ifnum#1>1 %
        \toks@=\expandafter{%
          \the\expandafter\toks@\expandafter,\expandafter\space
          \the\toks@@
        }%
      \fi
      \toks@=\expandafter{\the\toks@\space and #2}%
    \fi
    \else
      \ifcase#1 %
        \toks@={#2}%
        \@ReturnFiFi{%
          \scan@author1#3\@nil
        }%
      \else
        \ifnum#1>1 %
          \toks@=\expandafter{%
            \the\expandafter\toks@\expandafter,\expandafter\space
            \the\toks@@
          }%
      \fi
      \toks@@={#2}%
      \@ReturnFiFi{%
        \scan@author2#3\@nil
      }%
    \fi
  \fi
  }%
  \expandafter\expandafter\expandafter\scan@author
  \expandafter\expandafter\expandafter0%
  \expandafter\@author\space\and\@nil
  \edef\x{\endgroup
  \noexpand\hypersetup{pdfauthor={\the\toks@}}%
  }%
  \x
  }
\makeatother


\newcommand{\dst}{\varepsilon}
\newcommand{\ab}{k}
\newcommand{\ns}{n}
\newcommand{\dims}{d}

\newcommand{\p}{\mathbf{p}}
\newcommand{\q}{\mathbf{q}}

\usepackage{filecontents}

\title{Lest I forget it: $\lp[\infty]$ mean estimation of high-dimensional distributions}
\date{September, 2020}

\begin{document}
\begin{flushleft}\sf\footnotesize
\makeatletter
\@date~- \today \hfill \@title
\makeatother
\end{flushleft}
\vspace{5mm}

The goal of this short note is to record the following lower bounds on $\lp[\infty]$ mean estimation of $\dims$-dimensional spherical Gaussians and product binary distributions:
\begin{theorem}
  For $\dims\geq 1$ and $\dst\in(0,1]$, estimating the mean of (i)~an identity-covariance $\dims$-dimensional Gaussian and (ii)~a product distribution over $\{-1,1\}^\dims$ to $\lp[\infty]$ distance $\dst$ both has sample complexity $\bigTheta{\frac{\log\dims}{\dst^2}}$.
\end{theorem}
For convenience, we denote by $\operatorname{Rad}(\lambda)$ the distribution over $\{-1,1\}$ with expectation $\lambda\in[-1,1]$, and let $\mathcal{G}_\dims \eqdef \setOfSuchThat{\gaussian{\mu}{I_\dims}}{\mu\in\R^\dims}$ and $\mathcal{B}_\dims \eqdef \setOfSuchThat{ \otimes_{i=1}^\dims \operatorname{Rad}(\mu_i) }{\mu\in[-1,1]^\dims}$ be the families of identity-covariance Gaussians and product binary distributions considered, respectively.

\paragraph{Upper bound.} The upper bound, for both $\mathcal{G}_\dims$ and $\mathcal{B}_\dims$, follow from the following simple scheme: given $\ns=\bigO{(\log\dims)/\dst^2}$ i.i.d.\ samples from an unknown $\p\in\mathcal{G}_\dims$ (resp., $\p\in\mathcal{B}_\dims$), we can estimate independently the mean $\mu_i$ of each marginal to an additive $\dst$, with probability at least $1-\frac{1}{3\dims}$, by using the corresponding coordinate of the $\ns$ samples. By a union bound over all $\dims$ coordinates, the resulting $\hat{\mu}\in\R^\dims$ satisfies $\norminf{\hat{\mu}-\mu} \leq \dst$ with probability at least $1/3$.

\paragraph{Lower bound.} The lower bound is the more interesting part, and shows that this very naive approach (independently deal with each coordinate, then apply a union bound) is essentially the best one can do.

The following argument, which applies to both $\mathcal{G}_\dims$ and $\mathcal{B}_\dims$, was communicated to me by \href{https://people.ece.cornell.edu/acharya/}{Jayadev Acharya}. It shows the even stronger statement about \emph{testing} the mean of a Gaussian or product binary distribution, even under the promise that this mean is \emph{$1$-sparse}.
\begin{lemma}[Gaussian Hide-and-Seek]
  \label{lemma:hs:gaussian}
  For $\dims\geq 1$ and $\dst\in(0,1]$, distinguishing between $\gaussian{0}{I_\dims}$ and $\gaussian{\mu}{I_\dims}$ \emph{where $\mu$ is promised to satisfy $\mu=\dst e_i$ for some $i\in[\dims]$} requires $\bigOmega{\frac{\log\dims}{\dst^2}}$ samples.
\end{lemma}
\begin{proof}
  Fix any number of samples $\ns$. Denote by $\p^{(\ns)}$ the $\ns$-fold product of the standard Gaussian $\gaussian{0}{I_\dims}^{\otimes \ns} = \gaussian{0}{I_{\ns\dims}}$, and by $\q^{(\ns)}$ the uniform mixture $\q^{(\ns)} = \frac{1}{\dims}\sum_{i=1}^\dims \gaussian{\dst e_i}{I_\dims}^{\otimes \ns}$.  By a standard Le Cam-type argument, any $\ns$-sample test for the original problem can be used to distinguish $\p^{(\ns)}$ and $\q^{(\ns)}$, which is only possible if $\totalvardist{\p^{(\ns)}}{\q^{(\ns)}} \gtrsim 1$. Since
  \[
      \totalvardist{\p^{(\ns)}}{\q^{(\ns)}}^2 \leq \frac{1}{4}\chisquare{\q^{(\ns)}}{\p^{(\ns)}}
  \]
  it suffices to bound $\chisquare{\q^{(\ns)}}{\p^{(\ns)}}$. Which is convenient, as we can invoke~\autoref{lem:mixture_chisquare} to compute this explicitly. Indeed, for any $j\in[\ns]$ and any $i\in[\dims]$, the corresponding $\delta_j^i$ is independent of $j$ (as all marginals are the same) and equal to
  \[
      \delta_j^i(x) = \frac{\gaussian{\dst e_i}{I_\dims}(x)-\gaussian{0}{I_\dims}(x)}{\gaussian{0}{I_\dims}(x)} = e^{-\frac{\dst^2}{2}}e^{\dst x_i}-1,\,\quad x\in\R^\dims,
  \]
  so that for any two $i_1,i_2$, the $H_j(i_1,i_2)$ of~\autoref{lem:mixture_chisquare} (again independent of $j$) is equal to
  \[
      H_j(i_1,i_2) = \shortexpect_{X\sim\gaussian{0}{I_\dims}}[ (e^{-\frac{\dst^2}{2}}e^{\dst X_{i_1}}-1)(e^{-\frac{\dst^2}{2}}e^{\dst X_{i_2}}-1) ] = (e^{\dst^2}-1)\indic{i_1=i_2}\,.
  \]
  This is great as now we get from~\autoref{lem:mixture_chisquare} that
  \[
      \chisquare{\q^{(\ns)}}{\p^{(\ns)}} = \shortexpect_{i_1,i_2}[ (1+(e^{\dst^2}-1)\indic{i_1=i_2})^\ns ] -1 = \frac{\dims-1}{\dims}+\frac{e^{\ns\dst^2}}{\dims}-1 = \frac{e^{\ns\dst^2}-1}{\dims}
  \]
  and for this to be $\Omega(1)$ we need $\ns = \bigOmega{\frac{\log\dims}{\dst^2}}$.
\end{proof}
This was for Gaussians though. What about product binary distributions? As it turns out, the same argument goes through, nearly unchanged.
\begin{lemma}[Bernoullli Hide-and-Seek]
  \label{lemma:hs:bernoulli}
  For $\dims\geq 1$ and $\dst\in(0,1]$, distinguishing between the uniform distribution $\operatorname{Rad}(0)^{\otimes \dims}$ and $\otimes_{i=1}^\dims \operatorname{Rad}(\mu_i)$ \emph{where $\mu$ is promised to satisfy $\mu=\dst e_i$ for some $i\in[\dims]$} requires $\bigOmega{\frac{\log\dims}{\dst^2}}$ samples.
\end{lemma}
\begin{proof}
  Fix any number of samples $\ns$. Denote by $\p^{(\ns)}$ the $\ns$-fold product of the uniform distribution, $(\operatorname{Rad}(0)^{\otimes \dims})^{\otimes \ns} = \operatorname{Rad}(0)^{\otimes \ns\dims}$, and by $\q^{(\ns)}$ the uniform mixture $\q^{(\ns)} = \frac{1}{\dims}\sum_{i=1}^\dims \frac{1}{2^{\dims-1}}\operatorname{Rad}(\dst)$.\footnote{Note that for $\mu=\dst e_i$, we have $\otimes_{i=1}^\dims \operatorname{Rad}(\mu_i) = \frac{1}{2^{\dims-1}}\operatorname{Rad}(\dst)$ as all coordinates are uniform except one.}  As in the proof of~\autoref{lemma:hs:gaussian}, it suffices to bound $\chisquare{\q^{(\ns)}}{\p^{(\ns)}}$, and to do so we will compute it explicitly using~\autoref{lem:mixture_chisquare}. Indeed, for any $j\in[\ns]$ and any $i\in[\dims]$, the corresponding $\delta_j^i$ is independent of $j$ and can be seen to be equal to
  \[
      \delta_j^i(x) = \frac{\frac{1}{2^{\dims-1}}\operatorname{Rad}(\dst)(x)-\frac{1}{2^{\dims}}}{\frac{1}{2^{\dims}}} = \dst x_i,\,\quad x\in\R^\dims,
  \]
  so that for any two $i_1,i_2$, $H_j(i_1,i_2) = \dst^2\shortexpect_{X}[X_{i_1}X_{i_2} ] = \dst^2\indic{i_1=i_2}$ (where the expectation is over $X$ u.a.r. from $\{-1,1\}^\dims$).
  From~\autoref{lem:mixture_chisquare}, it follows that
  \[
      \chisquare{\q^{(\ns)}}{\p^{(\ns)}} = \shortexpect_{i_1,i_2}[ (1+\dst^2\indic{i_1=i_2})^\ns ] -1 
      %= \frac{\dims-1}{\dims}+\frac{(1+\dst^2)^\ns}{\dims}-1 
      = \frac{(1+\dst^2)^\ns-1}{\dims}
  \]
  and for this to be $\Omega(1)$ we again need $\ns = \bigOmega{\frac{\log\dims}{\dst^2}}$.
\end{proof}

We finally state the (relatively standard lemma) which allowed us to easily handle the chi square distance between a mixture and a product distribution.
\begin{lemma}[{See, e.g.,~\cite[Lemma III.5]{AcharyaCT18}}]\label{lem:mixture_chisquare}
Consider a random variable $\theta$ such that for each
$\theta=\vartheta$ the distribution $Q_\vartheta^\ns$ is defined as
$Q_{1,\vartheta} \times \dots \times Q_{n,\vartheta}$. Further, let
$P^\ns = P_1 \times \dots \times P_\ns$ be a fixed product
distribution. Then,
\[
\chi^2(\shortexpect_{\theta}[Q_\theta^\ns], P^\ns) = \shortexpect_{\theta\theta'}{\mleft[\prod_{j=1}^\ns (1+{H_j(\theta,\theta')})\mright]} - 1,
\]
where $\theta'$ is an independent copy of $\theta$, and with
$\delta_j^\vartheta(X_j) = (Q_{j,\vartheta}(X_j)-P_j(X_j))/P_j(X_j)$,
\[
H_j(\vartheta,\vartheta') \eqdef \dotprod{\delta_j^\vartheta}{\delta_j^{\vartheta'}}=\expect{\delta_j^\vartheta(X_j)\delta_j^{\vartheta'}(X_j)},
\]
where the expectation is over $X_j$ distributed according to $P_j$.
\end{lemma}

%%%%%%%%%% Bibliography
\begin{filecontents}{references-mean-estimation-linfty.bib}
@article{AcharyaCT18,
  author    = {Jayadev Acharya and
               Cl{\'{e}}ment L. Canonne and
               Himanshu Tyagi},
  title     = {Inference under Information Constraints {I:} Lower Bounds from Chi-Square
               Contraction},
  journal   = {CoRR},
  volume    = {abs/1812.11476},
  year      = {2018},
  url       = {http://arxiv.org/abs/1812.11476},
  archivePrefix = {arXiv},
  eprint    = {1812.11476},
  timestamp = {Wed, 02 Jan 2019 14:40:18 +0100},
  biburl    = {https://dblp.org/rec/journals/corr/abs-1812-11476.bib},
  bibsource = {dblp computer science bibliography, https://dblp.org}
}

\end{filecontents}
%%%%%%%%%%%%%%%%%%%%%%%%%%%%%%%%%%%%%%%%%%%%%%%%%%%%%%%%%%%%%%
\bibliographystyle{alpha}
\bibliography{references-mean-estimation-linfty}
\end{document}
