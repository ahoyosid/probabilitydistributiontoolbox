\documentclass[10pt]{article}
\def\withcolors{1}
\def\withnotes{1}
\def\withindex{0}
\usepackage[T1]{fontenc}
\usepackage[utf8]{inputenc}

%% Eye-candy
\usepackage{lmodern}
\usepackage{xspace}                                     % Smart spacing with \xspace
\usepackage[protrusion=true,expansion=true]{microtype}  % Improve font rendering

% Striking out text
\usepackage[normalem]{ulem}

%% Math
\usepackage{amsfonts,amsmath,amssymb, amsthm, mathtools}
\usepackage{thm-restate}
\usepackage{dsfont} % For the indicator symbol

% Algorithm environment
\usepackage{algorithmicx,algpseudocode,algorithm}

% Colors (with names)
\usepackage[usenames,dvipsnames,table]{xcolor}

% Quotes: \blockquote command
\usepackage{csquotes}

% Relative sizes for text
\usepackage{relsize}

% Bibliography
%\usepackage[numbers]{natbib}

% Required for the table of results
\usepackage{multirow}
\usepackage{chngpage} % allows for temporary adjustment of side margins

% For the commands such as \capitalisewords
\usepackage{mfirstuc}

% Graphics
\usepackage{tikz}
\usetikzlibrary{arrows}
\usetikzlibrary{calc,decorations.pathmorphing,patterns}

% For indexing
\ifnum\withindex=1
  \usepackage{makeidx}
  \usepackage{ifthen}
  \newcommand\indexed[2][]{\ifthenelse{\equal{#1}{}}{#2\index{#2}}{#2\index{#1}}}
  \makeindex %%%% Enable indexing
\fi
%%\usepackage{showidx} % To debug; does not play well with hyperref

% References and links
\usepackage[backref,colorlinks,citecolor=blue,bookmarks=true,linktocpage]{hyperref}
\usepackage{aliascnt}
\usepackage[numbered]{bookmark}

% Full pages
\usepackage{fullpage}

% Titling
\usepackage{titling}

% Compressed lists
\usepackage[shortlabels]{enumitem}
  \setitemize{noitemsep,topsep=3pt,parsep=2pt,partopsep=2pt} % Uncomment for compact item lists
  \setenumerate{itemsep=1pt,topsep=2pt,parsep=2pt,partopsep=2pt}
  \setdescription{itemsep=1pt}
  
% Package for todo notes.
\ifnum\withnotes=1
  \usepackage[colorinlistoftodos,textsize=scriptsize]{todonotes}
\fi

% Verbatim inputs and code
\usepackage{verbatim}

% Resizable parentheses that work (without the space between \left(#1\right)
\usepackage{mleftright} % \mleft( #1 \mright)

\makeatletter
\@ifundefined{theorem}{%
  % Theorems (each with its own style, all same counter). Cf. http://ftp.math.purdue.edu/mirrors/ctan.org/macros/latex/contrib/hyperref/doc/manual.pdf, p.17
  \theoremstyle{plain} %% Style
  	\newtheorem{theorem}{Theorem}%[section]
  	\newaliascnt{coro}{theorem}
  	  \newtheorem{corollary}[coro]{Corollary}
  	\aliascntresetthe{coro}
  	\newaliascnt{lem}{theorem}
  		\newtheorem{lemma}[lem]{Lemma}
  	\aliascntresetthe{lem}
  	\newaliascnt{clm}{theorem}
  		\newtheorem{claim}[clm]{Claim}
	\aliascntresetthe{clm}
	\newaliascnt{fact}{theorem}
 	 	\newtheorem{fact}[theorem]{Fact}
	\aliascntresetthe{fact}
  	\newtheorem*{unnumberedfact}{Fact}
  \newaliascnt{prop}{theorem}
  		\newtheorem{proposition}[prop]{Proposition}
	\aliascntresetthe{prop}
	\newaliascnt{conj}{theorem}
  		\newtheorem{conjecture}[conj]{Conjecture}
	\aliascntresetthe{conj}
  \theoremstyle{remark} %% Style
  	\newtheorem{remark}[theorem]{Remark}
  	\newtheorem{question}[theorem]{Question}
  	\newtheorem*{notation}{Notation}
 	 \newtheorem{example}[theorem]{Example}
  \theoremstyle{definition} %% Style
  	\newaliascnt{defn}{theorem}
 		 \newtheorem{definition}[defn]{Definition}
 	 \aliascntresetthe{defn}
}{}
\makeatother
\providecommand*{\lemautorefname}{Lemma} % For \autoref{} to know the name of lemmas
\providecommand*{\clmautorefname}{Claim}
\providecommand*{\propautorefname}{Proposition}
\providecommand*{\coroautorefname}{Corollary}
\providecommand*{\defnautorefname}{Definition}
\newenvironment{proofof}[1]{\begin{proof}[Proof of {#1}]}{\end{proof}}

%% \email{} command
\providecommand{\email}[1]{\href{mailto:#1}{\nolinkurl{#1}\xspace}}

%% Remarks and notes
\ifnum\withcolors=1
  \newcommand{\new}[1]{{\color{red} {#1}}} % new
  \newcommand{\newer}[1]{{\color{blue} {#1}}} % even newer
  \newcommand{\newest}[1]{{\color{orange} {#1}}} % even even newer
  \newcommand{\newerest}[1]{{\color{blue!10!black!40!green} {#1}}} % you get the idea.
  \newcommand{\ccolor}[1]{{\color{RubineRed}#1}} % Clement
\else
  \newcommand{\new}[1]{{{#1}}}
  \newcommand{\newer}[1]{{{#1}}}
  \newcommand{\newest}[1]{{{#1}}}
  \newcommand{\newerest}[1]{{{#1}}}
  \newcommand{\ccolor}[1]{{#1}}
\fi

\ifnum\withnotes=1
  \newcommand{\cnote}[1]{\par\ccolor{\textbf{C: }\sf #1}} % Clement
  \newcommand{\todonote}[2][]{\todo[size=\scriptsize,color=red!40,#1]{#2}}  
	\newcommand{\questionnote}[2][]{\todo[size=\scriptsize,color=blue!30]{#2}}
	\newcommand{\todonotedone}[2][]{\todo[size=\scriptsize,color=green!40]{$\checkmark$ #2}}
	\newcommand{\todonoteinline}[2][]{\todo[inline,size=\scriptsize,color=orange!40,#1]{#2}}  
  \newcommand{\marginnote}[1]{\todo[color=white,linecolor=black]{{#1}}}
\else
  \newcommand{\cnote}[1]{}
  \newcommand{\todonote}[2][]{\ignore{#2}}
	\newcommand{\questionnote}[2][]{\ignore{#2}}
	\newcommand{\todonotedone}[2][]{\ignore{#2}}
	\newcommand{\todonoteinline}[2][]{\ignore{#2}}
  \newcommand{\marginnote}[1]{\ignore{#1}}
\fi
\newcommand{\ignore}[1]{\leavevmode\unskip} % eat unnecessary spaces before
\newcommand{\cmargin}[1]{\questionnote{\ccolor{#1}}} % Clement

% Shortcuts
\newcommand{\eps}{\ensuremath{\varepsilon}\xspace}
\newcommand{\Algo}{\ensuremath{\mathcal{A}}\xspace} % Algorithm A
\newcommand{\Tester}{\ensuremath{\mathcal{T}}\xspace} % Testing algorithm T
\newcommand{\Learner}{\ensuremath{\mathcal{L}}\xspace} % Learning algorithm L
\newcommand{\property}{\ensuremath{\mathcal{P}}\xspace} % Property P
\newcommand{\class}{\ensuremath{\mathcal{C}}\xspace} % Class C
\newcommand{\eqdef}{\stackrel{\rm def}{=}}
\newcommand{\eqlaw}{\stackrel{\mathcal{L}}{=}}
\newcommand{\accept}{\textsf{ACCEPT}\xspace}
\newcommand{\fail}{\textsf{FAIL}\xspace}
\newcommand{\reject}{\textsf{REJECT}\xspace}
\newcommand{\opt}{{\textsc{opt}}\xspace}
\newcommand{\half}{\frac{1}{2}}
\newcommand{\domain}{\ensuremath{\Omega}\xspace} % Domain of a distribution (default notation)
\newcommand{\distribs}[1]{\Delta\!\left(#1\right)} % Domain of a distribution (default notation)
\newcommand{\yes}{{\sf{}yes}\xspace}
\newcommand{\no}{{\sf{}no}\xspace}
\newcommand{\dyes}{{\cal Y}}
\newcommand{\dno}{{\cal N}}

% Complexity
\newcommand{\littleO}[1]{{o\mleft( #1 \mright)}}
\newcommand{\bigO}[1]{{O\mleft( #1 \mright)}}
\newcommand{\bigOSmall}[1]{{O\big( #1 \big)}}
\newcommand{\bigTheta}[1]{{\Theta\mleft( #1 \mright)}}
\newcommand{\bigOmega}[1]{{\Omega\mleft( #1 \mright)}}
\newcommand{\bigOmegaSmall}[1]{{\Omega\big( #1 \big)}}
\newcommand{\tildeO}[1]{\tilde{O}\mleft( #1 \mright)}
\newcommand{\tildeTheta}[1]{\operatorname{\tilde{\Theta}}\mleft( #1 \mright)}
\newcommand{\tildeOmega}[1]{\operatorname{\tilde{\Omega}}\mleft( #1 \mright)}
\providecommand{\poly}{\operatorname*{poly}}

% Influence
\newcommand{\totinf}[1][f]{{\mathbf{Inf}[#1]}}
\newcommand{\infl}[2][f]{{\mathbf{Inf}_{#1}(#2)}}
\newcommand{\infldeg}[3][f]{{\mathbf{Inf}_{#1}^{#2}(#3)}}

% Sets and indicators
\newcommand{\setOfSuchThat}[2]{ \left\{\; #1 \;\colon\; #2\; \right\} } 			% sets such as "{ elems | condition }"
\newcommand{\indicSet}[1]{\mathds{1}_{#1}}                                              % indicator function
\newcommand{\indic}[1]{\indicSet{\left\{#1\right\}}}                                             % indicator function
\newcommand{\disjunion}{\amalg}%\coprod, \dotcup...

% Distance
\newcommand{\dtv}{\operatorname{d}_{\rm TV}}
\newcommand{\kl}{\operatorname{KL}}
\newcommand{\dhell}{\operatorname{d_{\rm{}H}}}
\newcommand{\hellinger}[2]{{\dhell\mleft({#1, #2}\mright)}}
\newcommand{\kldiv}[2]{{\kl\mleft({#1 \,\|\, #2}\mright)}}
\newcommand{\kolmogorov}[2]{{\operatorname{d_{\rm{}K}}\mleft({#1, #2}\mright)}}
\newcommand{\totalvardistrestr}[3][]{{\dtv^{#1}\mleft({#2, #3}\mright)}}
\newcommand{\totalvardist}[2]{\totalvardistrestr[]{#1}{#2}}
%\newcommand{\chisquarerestr}[3][]{{\operatorname{d}^{#1}_{\chi^2}\mleft({#2 \mid\mid #3}\mright)}}
\newcommand{\chisquare}[2]{{\chi^2\mleft({#1 \mid\mid #2}\mright)}}
\newcommand{\dist}[2]{\operatorname{dist}\mleft({#1, #2}\mright)}

% Restriction (functions, sequences, etc.)
\newcommand\restr[2]{{% we make the whole thing an ordinary symbol
  \left.\kern-\nulldelimiterspace % automatically resize the bar with \right
  #1 % the function
  \vphantom{\big|} % pretend it's a little taller at normal size
  \right|_{#2} % this is the delimiter
  }}

% Probability
\newcommand{\proba}{\Pr}
\newcommand{\probaOf}[1]{\proba\!\left[\, #1\, \right]}
\newcommand{\probaCond}[2]{\proba\!\left[\, #1 \;\middle\vert\; #2\, \right]}
\newcommand{\probaDistrOf}[2]{\proba_{#1}\left[\, #2\, \right]}

% Support of a distribution/function
\newcommand{\supp}[1]{\operatorname{supp}\!\left(#1\right)}

% Expectation & variance
\newcommand{\expect}[1]{\mathbb{E}\!\left[#1\right]}
\newcommand{\expectCond}[2]{\mathbb{E}\!\left[\, #1 \;\middle\vert\; #2\, \right]}
\newcommand{\shortexpect}{\mathbb{E}}
\newcommand{\var}{\operatorname{Var}}

% Distributions
\newcommand{\uniform}{\ensuremath{\mathcal{U}}}
\newcommand{\uniformOn}[1]{\ensuremath{\uniform\!\left( #1 \right) }}
\newcommand{\geom}[1]{\ensuremath{\operatorname{Geom}\!\left( #1 \right)}}
\newcommand{\bernoulli}[1]{\ensuremath{\operatorname{Bern}\!\left( #1 \right)}}
\newcommand{\bern}[2]{\ensuremath{\operatorname{Bern}^{#1}\!\left( #2 \right)}}
\newcommand{\binomial}[2]{\ensuremath{\operatorname{Bin}\!\left( #1, #2 \right)}}
\newcommand{\poisson}[1]{\ensuremath{\operatorname{Poisson}\!\left( #1 \right) }}
\newcommand{\gaussian}[2]{\ensuremath{ \mathcal{N}\!\left(#1,#2\right) }}
\newcommand{\gaussianpdf}[2]{\ensuremath{ g_{#1,#2}}}
\newcommand{\betadistr}[2]{\ensuremath{ \operatorname{Beta}\!\left( #1, #2 \right) }}

% Norms
\newcommand{\norm}[1]{\lVert#1{\rVert}}
\newcommand{\normone}[1]{{\norm{#1}}_1}
\newcommand{\normtwo}[1]{{\norm{#1}}_2}
\newcommand{\norminf}[1]{{\norm{#1}}_\infty}
\newcommand{\abs}[1]{\left\lvert #1 \right\rvert}
\newcommand{\dabs}[1]{\lvert #1 \rvert}
\newcommand{\dotprod}[2]{ \left\langle #1,\xspace #2 \right\rangle } 			% <a,b>
\newcommand{\ip}[2]{\dotprod{#1}{#2}} 			% shortcut

\newcommand{\vect}[1]{\mathbf{#1}} 			% shortcut

% Ceiling and floor
\newcommand{\clg}[1]{\left\lceil #1 \right\rceil}
\newcommand{\flr}[1]{\left\lfloor #1 \right\rfloor}

% Common sets
\newcommand{\R}{\ensuremath{\mathbb{R}}\xspace}
\newcommand{\C}{\ensuremath{\mathbb{C}}\xspace}
\newcommand{\Q}{\ensuremath{\mathbb{Q}}\xspace}
\newcommand{\Z}{\ensuremath{\mathbb{Z}}\xspace}
\newcommand{\N}{\ensuremath{\mathbb{N}}\xspace}
\newcommand{\cont}[1]{\ensuremath{\mathcal{C}^{#1}}}

% Oracles and variants
\newcommand{\ICOND}{{\sf INTCOND}\xspace}
\newcommand{\EVAL}{{\sf EVAL}\xspace}
\newcommand{\CDFEVAL}{{\sf CEVAL}\xspace}
\newcommand{\STAT}{{\sf STAT}\xspace}
\newcommand{\SAMP}{{\sf SAMP}\xspace}
\newcommand{\COND}{{\sf COND}\xspace}
\newcommand{\PCOND}{{\sf PAIRCOND}\xspace}
\newcommand{\ORACLE}{{\sf ORACLE}\xspace}

%% Terminology
\newcommand{\pdfsamp}{dual\xspace}
\newcommand{\cdfsamp}{cumulative dual\xspace}
\newcommand{\Pdfsamp}{\expandafter\capitalisewords\expandafter{\pdfsamp}}
\newcommand{\Cdfsamp}{\expandafter\capitalisewords\expandafter{\cdfsamp}}

% L_p norms
\newcommand{\lp}[1][1]{\ell_{#1}}

% Convolution
\DeclareMathOperator{\convolution}{\ast}

%% Terminology
\newcommand{\D}{\ensuremath{D}}
\newcommand{\distrD}{\ensuremath{\mathcal{D}}}
\newcommand{\birge}[2][\D]{\Phi_{#2}(#1)}
\newcommand{\iid}{i.i.d.\xspace}

% Sign
\DeclareMathOperator{\sign}{sgn}

%% Roman numerals
\makeatletter
\newcommand{\rom}[1]{\romannumeral #1}
\newcommand{\Rom}[1]{\expandafter\@slowromancap\romannumeral #1@}
\newcommand{\century}[2][th]{\Rom{#2}\textsuperscript{#1}}
\makeatother

% Hyperref and \autoref{} -- names
\renewcommand{\sectionautorefname}{Section} % To have "Section 5" instead of "section 5" with \autoref{}
\renewcommand{\chapterautorefname}{Chapter} % To have "Chapter 5" instead of "chapter 5" with \autoref{}
\renewcommand{\subsectionautorefname}{Section} % To have "Section 5" instead of "subsection 5" with \autoref{}
\renewcommand{\subsubsectionautorefname}{Section} % To have "Section 5" instead of "subsubsection 5" with \autoref{}
\def\algorithmautorefname{Algorithm}


%%%%%%%%%%%%%%%%%%%%%%%%%%%%%%%%%%%%%%%%%%%%%%%%%%%%%%%%%%%%%%%%%
% Add author and title info to PDF (and handles multiple authors)
%%%%%%%%%%%%%%%%%%%%%%%%%%%%%%%%%%%%%%%%%%%%%%%%%%%%%%%%%%%%%%%%%
\makeatletter
  \AtBeginDocument{
  \begingroup
  \toks@={}%
  \toksdef\toks@@=2 %
  \toks@@={}%
  \long\def\@ReturnFiFi#1#2\fi\fi{\fi\fi#1}%
  \def\scan@author#1#2 \and#3\@nil{%
  \ifx\\#3\\%
    \ifcase#1 %
      \toks@={#2}%
    \else
      \ifnum#1>1 %
        \toks@=\expandafter{%
          \the\expandafter\toks@\expandafter,\expandafter\space
          \the\toks@@
        }%
      \fi
      \toks@=\expandafter{\the\toks@\space and #2}%
    \fi
    \else
      \ifcase#1 %
        \toks@={#2}%
        \@ReturnFiFi{%
          \scan@author1#3\@nil
        }%
      \else
        \ifnum#1>1 %
          \toks@=\expandafter{%
            \the\expandafter\toks@\expandafter,\expandafter\space
            \the\toks@@
          }%
      \fi
      \toks@@={#2}%
      \@ReturnFiFi{%
        \scan@author2#3\@nil
      }%
    \fi
  \fi
  }%
  \expandafter\expandafter\expandafter\scan@author
  \expandafter\expandafter\expandafter0%
  \expandafter\@author\space\and\@nil
  \edef\x{\endgroup
  \noexpand\hypersetup{pdfauthor={\the\toks@}}%
  }%
  \x
  }
\makeatother


\newcommand{\p}{\mathbf{p}}
\newcommand{\q}{\mathbf{q}}
\newcommand{\bigp}{\mathbf{P}}
\newcommand{\bigq}{\mathbf{Q}}
\newcommand{\dst}{\varepsilon}
\newcommand{\ns}{n}



\title{A theorem of Valiant and Valiant}
\date{April, 2020}

\begin{filecontents}{testinglowerbound-useful.bib}
@article{GibbsS02,
  author = {Gibbs, Alison L. and Su, Francis Edward},
  title = {On Choosing and Bounding Probability Metrics},
  journal = {International Statistical Review},
  volume = {70},
  number = {3},
  pages = {419-435},
  doi = {10.1111/j.1751-5823.2002.tb00178.x},
  url = {https://onlinelibrary.wiley.com/doi/abs/10.1111/j.1751-5823.2002.tb00178.x},
  eprint = {https://onlinelibrary.wiley.com/doi/pdf/10.1111/j.1751-5823.2002.tb00178.x},
  year = {2002}
}

@inproceedings{DaskalakisKW17,
    AUTHOR = {Daskalakis, Constantinos and Kamath, Gautam and Wright, John},
     TITLE = {Which distribution distances are sublinearly testable?},
 BOOKTITLE = {Proceedings of the {T}wenty-{N}inth {A}nnual {ACM}-{SIAM}
              {S}ymposium on {D}iscrete {A}lgorithms},
     PAGES = {2747--2764},
 PUBLISHER = {SIAM, Philadelphia, PA},
      YEAR = {2018},
   MRCLASS = {68Q87 (62B10 62G10)},
  MRNUMBER = {3775963},
       DOI = {10.1137/1.9781611975031.175},
       URL = {https://doi.org/10.1137/1.9781611975031.175},
}

@article{ValiantV17,
    AUTHOR = {Valiant, Gregory and Valiant, Paul},
     TITLE = {An automatic inequality prover and instance optimal identity
              testing},
   JOURNAL = {SIAM J. Comput.},
  FJOURNAL = {SIAM Journal on Computing},
    VOLUME = {46},
      YEAR = {2017},
    NUMBER = {1},
     PAGES = {429--455},
      ISSN = {0097-5397},
   MRCLASS = {68Q30 (26D15 62G10 68Q25 68T15)},
  MRNUMBER = {3614697},
MRREVIEWER = {Gilles Teyssi\`ere},
       DOI = {10.1137/151002526},
       URL = {https://doi.org/10.1137/151002526},
      NOTE = {Journal version of~\cite{ValiantValiant14}.}
}

@incollection{ValiantValiant14,
    AUTHOR = {Valiant, Gregory and Valiant, Paul},
     TITLE = {An automatic inequality prover and instance optimal identity
              testing},
 BOOKTITLE = {55th {A}nnual {IEEE} {S}ymposium on {F}oundations of
              {C}omputer {S}cience---{FOCS} 2014},
     PAGES = {51--60},
 PUBLISHER = {IEEE Computer Soc., Los Alamitos, CA},
      YEAR = {2014},
   MRCLASS = {62E15 (94A15)},
  MRNUMBER = {3344854},
       DOI = {10.1109/FOCS.2014.14},
       URL = {https://doi.org/10.1109/FOCS.2014.14},
}

@InProceedings{AcharyaCT19,
  title = 	 {Inference under Information Constraints: Lower Bounds from Chi-Square Contraction},
  author = 	 {Acharya, Jayadev and Canonne, Cl{\'{e}}ment L and Tyagi, Himanshu},
  booktitle = 	 {Proceedings of the Thirty-Second Conference on Learning Theory},
  pages = 	 {3--17},
  year = 	 {2019},
  editor = 	 {Beygelzimer, Alina and Hsu, Daniel},
  volume = 	 {99},
  series = 	 {Proceedings of Machine Learning Research},
  address = 	 {Phoenix, USA},
  month = 	 {25--28 Jun},
  publisher = 	 {PMLR},
  pdf = 	 {http://proceedings.mlr.press/v99/acharya19a/acharya19a.pdf},
  url = 	 {http://proceedings.mlr.press/v99/acharya19a.html},
}

@misc{Pollard03,
  author = {Pollard, David},
  title = {Asymptopia},
  url = {http://www.stat.yale.edu/~pollard/Books/Asymptopia/},
  note = {Manuscript},
  year = 2003,
  urldate = {2016-11-08}
}

\end{filecontents} 

\begin{document}
\begin{flushleft}\sf\footnotesize
\makeatletter
\@date~- \today \hfill \@title
\makeatother
\end{flushleft}
\vspace{5mm}
The goal of this short document is to highlight a very useful result due to Valiant and Valiant~\cite{ValiantV17}, and slightly simplify a component of their proof (\autoref{lemma:hellinger:poisson}) by using the chi-squared distance instead of Hellinger. (We also discuss, in~\autoref{sec:hell:yeah:chisquare}, the advantage of the former over the latter.) Throughout, we write $\N$ for the set of non-negative integers, and $a\land b$ (resp. $a\lor b$) to denote the minimum (resp. maximum) of $a$ and $b$.

\begin{theorem}[{\cite[Theorem 4.2]{ValiantV17}}]
        \label{theo:vv:lb}
    Given a distribution $\p$ over $\N$, and associated values $\alpha_i$ such that $\alpha_i \in [0,1]$ for all $i\in\N$, define the distribution over distributions $Q$ by the following process: independently for each $i\in\N$, uniformly choose $z_i\in\{-1,1\}$, set $\tilde{\q}_i=(1+z_i\alpha_i)\p(i)$,and then normalize $\tilde{\q}$ to obtain a distribution $\q$. Then there exists an absolute constant $c>0$ such that it takes at least $c(\sum_i \alpha_i^4\p(i)^2)^{-1/2}$ samples to distinguish $\p$ from $Q$ with success probability $2/3$. Further, with probability at least $1/2$, the total variation distance between a random distribution from $Q$ and $\p$ is at least 
$
    \frac{1}{2}\min\left( \sum_{i\in\N} \alpha_i\p(i) - \max_{i} \alpha_i\p(i), \frac{1}{2}\sum_{i\in\N} \alpha_i\p(i) \right)\,.
$
\end{theorem}
\begin{proof}
The proof almost identically follows that of~\cite[Theorem 4.2]{ValiantV17}. We first argue the first part of the statement, about indistinguishability. Instead of considering directly $\p$ and the mixture $\bigq \eqdef \shortexpect_{\q\sim Q}[\q]$, we will, fixing a target number of samples $\ns$, focus on the related Poisson processes $\Pi_\p,\Pi_{\bigq}$ defined as follows:
\begin{itemize}
  \item $\Pi_\p$ is the product distribution $\bigotimes_{i=1}^\infty \poisson{\ns\p(i)}$ over $\N^{\N}$ (i.e., each coordinate $i$ is independent of all others, and is a $\poisson{\ns\p(i)}$ r.v.);
  \item $\Pi_{\bigq}$ is the mixture of product distributions $\shortexpect_{\q\sim Q}[\bigotimes_{i=1}^\infty \poisson{\ns\tilde{\q}(i)}]$ over $\N^{\N}$ (i.e., each coordinate $i$ is independent of all others and is a $\poisson{\ns(1+z_i\alpha_i)\p(i)}$ r.v., where $z_i$ is u.a.r. in $\{-1,1\}$).
\end{itemize}
We first argue that, given a draw from $\Pi_\p$ (resp. $\Pi_{\bigq}$), one can generate, with probability at least $1/2$, $\ns$ i.i.d.\ samples from $\p$ (resp. $\bigq$) as follows. Given a draw $\mathbf{x}\in\N^{\N}$ from either $\Pi_\p$ or $\Pi_{\bigq}$:
\begin{itemize}
  \item if $\sum_{i=1}^\infty \mathbf{x}_i < \ns$, then output \fail.
  \item Otherwise, create a (finite)\footnote{Note that $\sum_{i=1}^\infty \mathbf{x}_i$ is itself a $\poisson{\ns}$ random variable, so finite a.s.} multiset $T$ of length $\sum_{i=1}^\infty \mathbf{x}_i\geq \ns$ containing $\mathbf{x}_i$ occurrences of each $i\in\N$, and select u.a.r. a multiset $S$ of $T$ of size $\ns$. Output $S$.
\end{itemize}
It is not hard to see that, conditioned on not outputting $\fail$, the above process outputs a set of $\ns$ i.i.d.\ samples distributed exactly according to $\p$ (resp. $\bigq$). Thus, it suffices to show that the probability of outputting \fail is at most $1/2$. This itself is a consequence of the fact that $\sum_{i=1}^\infty \mathbf{x}_i\sim \poisson{\ns}$, so that, $\ns$ being an integer, its median is exactly $\ns$.

Suppose that, for a given $\ns$, one cannot distinguish $\Pi_\p$ and $\Pi_{\bigq}$ with advantage $1/12$; that is, given a sample drawn from either $\Pi_\p$ or $\Pi_{\bigq}$ (each with probability $1/2$), the probability to guess correctly which one is less than $1/2+1/12=7/12$: then we claim that, given $\ns$ samples, one cannot distinguish $\p$ from $\bigq$ with advantage $1/6$. Indeed, by contradiction, suppose we have an tester $\Tester$ for the latter task with sample complexity $\ns$ and advantage at least $1/6$:
\begin{equation}
      \probaDistrOf{ \substack{b\sim\bernoulli{1/2}\\ \mathbf{h}\sim b\p+(1-b)\bigq } }{ \Tester, \text{ given }\ns\text{ samples from }\mathbf{h},\text{ outputs } b } \geq \frac{2}{3}\,.
\end{equation}
Then we can get a distinguisher $\Tester'$ between $\Pi_\p$ and $\Pi_{\bigq}$ with advantage $1/12$: given a sample a draw $\mathbf{x}\in\N^{\N}$ from either $\Pi_\p$ or $\Pi_{\bigq}$, $\Tester'$ tries to generate $\ns$ i.i.d.\ samples as described above. If the output is \fail, then it outputs a bit uniformly at random; otherwise, it runs $\Tester$ on the resulting $\ns$ samples and outputs what $\Tester$ returns. It follows that
\begin{equation}
      \probaDistrOf{ \substack{b\sim\bernoulli{1/2}\\ \mathbf{x}\sim b\Pi_\p+(1-b)\Pi_\bigq } }{ \Tester', \text{ given } \mathbf{x},\text{ outputs } b }
      \geq \frac{1}{2}\probaOf{\fail} + \frac{2}{3}(1-\probaOf{\fail})      
       \geq \frac{7}{12}\,,
\end{equation}
a contradiction. Therefore, it suffices that one cannot distinguish $\Pi_\p$ and $\Pi_{\bigq}$ with advantage $1/12$; for which it is enough to show that unless $\ns$ is large enough, the total variation distance $\totalvardist{\Pi_\p}{\Pi_{\bigq}}$ is smaller than some absolute constant $c>0$. The crux is that, due to the definition of our two processes, both $\Pi_\p$ and $\Pi_{\bigq}$ are product distributions. Therefore, using the subadditivity of Hellinger distance for product distributions, we can now bound
\begin{equation}
      \totalvardist{\Pi_\p}{\Pi_{\bigq}}^2 \leq \hellinger{\Pi_\p}{\Pi_{\bigq}}^2 \leq \sum_{i\in\N}\hellinger{ \poisson{\ns\p(i)} }{ \Pi_{\bigq,i} }^2
\end{equation}
where $\Pi_{\bigq,i} = \frac{1}{2}\left( \poisson{\ns(1+\alpha_i)\p(i)}+\poisson{\ns(1-\alpha_i)\p(i)} \right)$. This is where we invoke~\autoref{lemma:hellinger:poisson}, leading to
\begin{equation}
      \totalvardist{\Pi_\p}{\Pi_{\bigq}}^2 \leq \sum_{i\in\N} \alpha_i^4 (\ns\p_i)^2 = \ns^2 \sum_{i\in\N} \alpha_i^4\p_i^2\,.
\end{equation}
For the total variation to be at least $c$, we thus need $\ns \geq c^{1/2}\left(\sum_{i\in\N} \alpha_i^4\p_i^2\right)^{-1/2}$, concluding the proof of indistinguishability.\smallskip

We now turn to the second part of the claim about the distance. To ease notation, set $w_i \eqdef \alpha_i\p(i)$ for all $i\in\N$, and assume without loss of generality that the sequence $w$ is non-increasing; our goal is then to show 
\[
    \probaDistrOf{\q\sim Q}{ \totalvardist{\p}{\q} \geq \frac{1}{2}\min( \normone{w} - \norminf{w}, \frac{1}{2}\normone{w}) } \geq \frac{1}{2}\,.
\] 
Observe that for any $\q$ (defined from the corresponding sequence $z\in\N^{\N}$), we have, since $\q = \tilde{\q}/\sum_i \tilde{\q}(i)$ and $\sum_i \tilde{\q}(i) = 1+\sum_{i} z_iw_i$,
\begin{equation}
    2\totalvardist{\p}{\q} = \sum_{i\in\N} \abs{ \p(i) - \q(i) } \geq \sum_{i\in\N} \abs{ \p(i) - \tilde{\q}(i) } - \sum_{i\in\N} \abs{ \tilde{\q}(i) - \q(i) }
    = \normone{w} - \dabs{\sum_{i\in\N} z_iw_i}\,.
\end{equation}
Therefore, it suffices to show that $\dabs{\sum_{i\in\N} z_i w_i} \leq \norminf{w} \lor \frac{1}{2}\normone{w}$ with probability at least $1/2$. We proceed by a distinction of cases: first, suppose $w_0 = \norminf{w} \geq \frac{1}{2}\normone{w}$. Then 
\[
    \probaOf{ \Big| \sum_{i\geq 0} z_i w_i \Big| \leq \norminf{w} } = \probaOf{ \Big| z_0\norminf{w} + \sum_{i\geq 1} z_i w_i \Big| \leq \norminf{w} } = 
    \probaOf{ \sum_{i\geq 1} z_i w_i \leq 0 } = \frac{1}{2}
\]
by symmetry.

 Otherwise, assume $\norminf{w} < \frac{1}{2}\normone{w}$, and consider the index $t\geq 1$ such that $\sum_{i=0}^{t-1} w_i \leq \frac{1}{2}\normone{w} < \sum_{i=0}^{t} w_i$. Note that this implies $\sum_{i=0}^{t-1} w_i-\sum_{i=t}^{\infty} w_i \geq -\frac{1}{2}\normone{w}$, as otherwise we could write
 \[
    \normone{w} \geq w_0 + \sum_{i=t}^{\infty} w_i \geq a_t + \sum_{i=t}^{\infty} w_i > a_t + \sum_{i=0}^{t-1} w_i + \frac{1}{2}\normone{w} > \normone{w}\,,
\]
 a contradiction. Then, since $\sum_{i=0}^{t-1} z_i w_i$ and $\sum_{i=t}^\infty z_i w_i$ have opposite signs with probability $1/2$,
 \[
    \probaOf{ \Big| \sum_{i\geq 0} z_i w_i \Big| \leq \frac{1}{2}\normone{w} } = \probaOf{ \Big| \sum_{i=0}^{t-1} z_i w_i + \sum_{i=t}^\infty z_i w_i \Big| \leq \frac{1}{2}\normone{w} } \geq \frac{1}{2}
\]
concluding the proof.
\end{proof}
%%%%%%%%%%%%%%%%%%%%%%%%%%%%%%%%%%%%%%%%%%%%%%%%%%%%%%%%%%%%%
\section{Bounding distances between a Poisson and a mixture}
\begin{lemma}[Hellinger bound]
  \label{lemma:hellinger:poisson}
Let $\lambda>0$, and $\alpha\in[0,1]$. Define $\bigq\eqdef \frac{1}{2}(\poisson{(1+\alpha)\lambda} +\poisson{ (1-\alpha)\lambda) }$. Then $\hellinger{ \poisson{\lambda} }{ \bigq } \leq \alpha^2\lambda$.
\end{lemma}
(Note that proving the quadratically weaker bound $\hellinger{ \poisson{\lambda} }{ \bigq }^2 \lesssim \alpha^2\lambda$ is straightforward, but insufficient to our purposes.) \autoref{lemma:hellinger:poisson} will follow from the analogous (but easier to prove) claim for chi-squared distance, along with~\autoref{fact:hellinger:chisquare}. Note that as the chi-squared distance is not symmetric, the order of the distributions matters in~\autoref{lemma:chisquare:poisson}.
\begin{lemma}[$\chi^2$ bound]
  \label{lemma:chisquare:poisson}
Let $\lambda>0$, and $\alpha\in[0,1]$. Define $\bigq\eqdef \frac{1}{2}(\poisson{(1+\alpha)\lambda} +\poisson{ (1-\alpha)\lambda) }$. Then $1\land \chisquare{ \bigq }{ \poisson{\lambda} } \leq \alpha^4\lambda^2$.
\end{lemma}
\begin{proof}
  We can assume in the rest of the proof that $\alpha^2\lambda \leq 1$, as otherwise there is nothing to prove (the LHS being at most $1$ due to the minimum). For convenience, write $\bigp\eqdef \poisson{\lambda}$. We can express the pmf of $\bigq$ as 
  \begin{equation}
      \bigq(n) = \frac{1}{2}\left( e^{-\lambda(1+\alpha)}\frac{\lambda^n(1+\alpha)^n}{n!} + e^{-\lambda(1-\alpha)}\frac{\lambda^n(1-\alpha)^n}{n!} \right)
      = \bigp(n)\cdot\frac{e^{-\lambda\alpha}(1+\alpha)^n + e^{\lambda\alpha}(1-\alpha)^n}{2}
  \end{equation}
  for $n\in \N$. It follows that
  \begin{equation}
    \label{eq:expr:chisquare}
      \chisquare{ \bigq }{ \bigp } = -1 + \sum_{n\in\N} \frac{\bigq(n)^2}{\bigp(n)} = -1 + e^{-\lambda}\sum_{n\in\N} \frac{\lambda^n}{n!}\left( \frac{e^{-\lambda\alpha}(1+\alpha)^n + e^{\lambda\alpha}(1-\alpha)^n}{2}\right)^2
  \end{equation}
  Focusing on the last sum, we expand the square and compute it explicitly:
  \begin{align*}
  \sum_{n\in\N} \frac{\lambda^n}{n!}\left( \frac{e^{-\lambda\alpha}(1+\alpha)^n + e^{\lambda\alpha}(1-\alpha)^n}{2}\right)^2
  &= \sum_{n\in\N} \frac{\lambda^n}{n!} \frac{e^{-2\lambda\alpha}(1+\alpha)^{2n} + e^{2\lambda\alpha}(1-\alpha)^{2n} + 2(1-\alpha^2)^n}{4}\\
  &= \frac{e^{-2\lambda\alpha}e^{\lambda(1+\alpha)^{2}} + e^{2\lambda\alpha}e^{\lambda(1-\alpha)^{2}} + 2e^{\lambda(1-\alpha^2)}}{4}\\
  &= e^{\lambda}\cdot\frac{e^{\lambda\alpha^2} + e^{-\lambda\alpha^2}}{2}\,.
  \end{align*}
  Plugging this in~\eqref{eq:expr:chisquare}, we get
  \begin{equation}
      \chisquare{ \bigq }{ \bigp } = -1 + \frac{e^{\lambda\alpha^2} + e^{-\lambda\alpha^2}}{2} \leq \lambda^2\alpha^4
  \end{equation}
  where for the last inequality we used our bound $\lambda\alpha^2\leq 1$, and the fact that $\cosh x \leq 1+x^2$ for $|x|\leq 1$. This concludes the proof.
\end{proof}

%%%%%%%%%%%%%%%%%%%%%%%%%%%%%%%%%%%%%%%%%%%%%%%%%%%%%%%%%%%%%
\section{Why $\chi^2$ instead of Hellinger}\label{sec:hell:yeah:chisquare}
At first glance, our choice to use the chi-squared distance instead of Hellinger distance for the key technical lemma may seem peculiar. After all, the chi-squared distance (which is, at the end of the day, merely a first-order approximation to the Kulllback--Leibler divergence\footnote{Indeed, for the KL divergence in nats,
\[
    \kldiv{\p}{\q} = \sum_{i} \p(i)\ln\frac{\p(i)}{\q(i)} = \sum_{i} \p(i)\ln\mleft(1+\frac{\p(i)-\q(i)}{\q(i)} \mright) \approx \sum_{i} \frac{\p(i)^2}{\q(i)} - 1 = \chisquare{\p}{\q}
\]
since $\ln(1+x) \approx x$ and $\chisquare{\p}{\q} = \sum_{i} \frac{(\p(i)-\q(i))^2}{\q(i)} = \sum_{i} \frac{\p(i)^2}{\q(i)} - 1$.
}) is not bounded, and not even a distance (being asymmetric); while the Hellinger distance is bounded, behaves nicey with respect to product distributions (e.g., via subadditivity of its square), and overall looks clean and appealing.

However, a pervasice technique in proving sample complexity lower bounds involves reference distribution $\p$ and a family of \emph{perturbations} of $\p$, $(\p_z)_{z\in\mathcal{Z}}$ (for some suitable parameter set $\mathcal{Z}$), such that $\p_z(i) = (1+\delta(i,z))\p(i)$ for all $i$. The key then is to upper bound the total variation distance between the reference distribution and the \emph{mixture} of perturbations,
\[
    \totalvardist{ \p }{ \shortexpect_Z[\p_Z] }
\]
(instead of the looser $\shortexpect_Z[ \totalvardist{ \p }{ \p_Z } ]$, which lacks a lot of useful cancellations and is typically much bigger). But using the Hellinger distance as a proxy will then involve a rather nasty square root: even assuming that $\bigq \eqdef \shortexpect_Z[\p_Z]$
\[
    \hellinger{ \p }{ \shortexpect_Z[\p_Z] }^2 = \sum_i \left( \sqrt{\p(i)} - \sqrt{\shortexpect_Z[\p_Z(i)]} \right)^2
    = \sum_i \p(i)\left( 1 - \sqrt{1+\shortexpect_Z[\delta(i,Z)]} \right)^2
\]
which is generally \emph{not} a fun task. Yet, bounding the total variation distance by the (now bounded) quantity $1\land \chisquare{ \shortexpect_Z[\p_Z] }{ \p }$ leads to an expression of the form
\[
    \chisquare{ \shortexpect_Z[\p_Z] }{ \p } = \sum_i \frac{\left( \shortexpect_Z[\p_Z(i)] - \p(i) \right)^2}{\p(i)}
    = \sum_i \p(i)\shortexpect_Z[\delta(i,Z)]^2
\]
(the order of the distributions in the chi-squared distance will typically matter a lot: mixture first). Squares are in my experience nicer to handle than square roots. 

Further, we have many other tools to deal with the chi-squared distance; for instance, the handy lemma below, due to~\cite{Pollard03}, which enables us to handle chi-square distances of mixtures with respect to a reference product distribution.
\begin{lemma}[{\cite[Lemma~8]{AcharyaCT19}}]\label{lem:mixture_chisquare}
Consider a random variable $Z$ such that for each
$Z=z$ the distribution $\q_z^\ns$ is defined as
$\q_{1,z} \times \dots \times \q_{\ns,z}$. Further, let
$\p^\ns = \p_1 \times \dots \times \p_\ns$ be a fixed product
distribution. Then,
\[
\chisquare{ \shortexpect_Z[\q_Z^\ns] }{ \p^\ns } = \shortexpect_{ZZ'}[ \prod_{j=1}^\ns (1+{\Delta_j(Z,Z')}) ] - 1,
\]
where $Z'$ is an independent copy of $Z$, and with
$\delta_j^z(x_j) = \frac{\q_{j,z}(x_j)-\p_j(x_j)}{\p_j(x_j)}$, $\Delta_j(z,z')$ is the \emph{chi-squared correlation}
\[
\Delta_j(z,z') \eqdef \expect{\delta_j^z(X_j)\delta_j^{z'}(X_j)},
\]
where the expectation is over $X_j$ distributed according to $\p_j$.
\end{lemma}

\appendix

%%%%%%%%%%%%%%%%%%%%%%%%%%%%%%%%%%%%%%%%%%%%%%%%%%%%%%%%%%%%%
\section{Bounding Hellinger by $\chi^2$}
For the sake of self-completeness, we give a simple proof of the fact that the chi-squared distance upper bounds the squared Hellinger one.
\begin{fact}
  \label{fact:hellinger:chisquare}
For any two discrete distributions $\p_1,\p_2$, $\hellinger{\p_1}{\p_2}^2 \leq 1\land \frac{1}{2}\chisquare{\p_1}{\p_2}$.
\end{fact}
\begin{proof}
This is easily shown from (i)~$\hellinger{\p_1}{\p_2}\leq 1$, and (ii)~the identity $a-b = (\sqrt{w}-\sqrt{b})(\sqrt{w}+\sqrt{b})$, as
\[
    2\hellinger{\p_1}{\p_2}^2 = \sum_i \Big(\sqrt{\p_1(i)}-\sqrt{\p_2(i)}\Big)^2 = \sum_i \frac{(\p_1(i)-\p_2(i))^2}{\Big(\sqrt{\p_1(i)}+\sqrt{\p_2(i)}\Big)^2} \leq \sum_i \frac{(\p_1(i)-\p_2(i))^2}{\p_2(i)}\,,
\]
which is exactly $\chisquare{\p_1}{\p_2}$.
See, e.g.~\cite{GibbsS02} for the continuous case as well, or \cite[Proposition~1]{DaskalakisKW17} for a generalization to subdistributions.
\end{proof}

%%%%%%%%%%%%%%%%%%%%%%%%%%%%%%%%%%%%%%%%%%%%%%%%
\bibliographystyle{alpha}
\bibliography{testinglowerbound-useful}
\end{document}
